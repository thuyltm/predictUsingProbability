\documentclass[a4paper, 13pt]{report}
\usepackage{comment} % enables the use of multi-line comments (\ifx \fi) 
\usepackage{lipsum} %This package just generates Lorem Ipsum filler text. 
\usepackage{fullpage} % changes the margin
\usepackage[utf8]{vietnam}
\usepackage{amsmath}
\usepackage{graphicx}
\graphicspath{ {images/} }
\usepackage{caption}
\usepackage{enumitem}
\usepackage{array}
\newcommand\tab[1][1cm]{\hspace*{#1}}
\newenvironment{steps}[1]{\begin{enumerate}[label=#1 \arabic*]}{\end{enumerate}}
\usepackage{placeins}
\usepackage{pbox}
\newcommand\SLASH{\char`\\}
\usepackage{fancybox}
\usepackage{placeins}
\usepackage[document]{ragged2e}
\usepackage{relsize}
\usepackage{amssymb}
\usepackage{ctable} % for \specialrule command
\usepackage{multirow}
\usepackage{wrapfig}
\usepackage{subcaption}
\setcounter{tocdepth}{1} % for \tableofcontents


\makeatletter% http://tex.stackexchange.com/questions/29517/forcing-new-line-after-item-number-in-enumerate-environment/29518#29518
\def\step{%
   \@ifnextchar[ \@step{\@noitemargtrue\@step[\@itemlabel]}}
\def\@step[#1]{\item[#1]\mbox{}}
\makeatother

\begin{document}

\newpage
\pagenumbering{gobble}  % ------- bỏ đánh số trang----------------
% trang bìa 1-----------------------------------------------------
	%\thispagestyle{empty}
	\thisfancypage{
		\doublebox}
	{}
	
	\begin{center}
		\vspace*{0.2cm}
		ĐẠI HỌC QUỐC GIA TP. HỒ CHÍ MINH\\
		{\fontsize{14pt}{1}\textbf{		TRƯỜNG ĐẠI HỌC BÁCH KHOA }}\\
		-------------------------------\\
		\includegraphics[width=3cm]{images/bk}\\
		\vspace*{2cm}
		{\fontsize{14pt}{1}\selectfont LÊ THỊ MINH THÙY}\\
		\vspace*{2cm}
		{\fontsize{16pt}{1}\selectfont \textbf{DỰ ĐOÁN XÁC SUẤT \\XE BUÝT VỀ TRẠM ĐÚNG GIỜ}}\\
		\vspace*{2cm}
		\begin{flushleft}
		\large
		\begin{tabular}{ ll } 
        \hspace{1cm} Chuyên ngành: & Khoa học máy tính\\
		\hspace{1cm} Mã số: & 60.48.01\\
		\end{tabular}
		\end{flushleft}
		\vspace*{2cm}
		{\fontsize{14pt}{1}\selectfont LUẬN VĂN THẠC SĨ}\\
		\vspace*{2cm}
		{\fontsize{12pt}{1}	TP.Hồ Chí Minh, \today}\\
		\vspace*{2cm}
		{\fontsize{12pt}{1}	Công trình được hoàn thành tại:\\ Trường Đại Học Bách Khoa - ĐHQG - TPHCM}
	\end{center}
%----------------------------------------------------------------------------------------	
\pagebreak
Công trình được hoàn thành tại: \textbf{Trường Đại Học Bách Khoa - ĐHQG - TPHCM}\\
Cán bộ hướng dẫn khoa học: TS. Huỳnh Tường Nguyên	\\
	(Ghi rõ họ, tên, học hàm, học vị và chữ  ký)\\

\vspace*{4cm}Cán bộ chấm nhận xét 1: 	\\
	(Ghi rõ họ, tên, học hàm, học vị và chữ  ký)\\

\vspace*{4cm}Cán bộ chấm nhận xét 2:	\\
	(Ghi rõ họ, tên, học hàm, học vị và chữ  ký)\\

\vspace*{4cm}Luận văn thạc sĩ được bảo vệ tại Trường Đại học Bách Khoa, ĐHQG Tp. HCM \\
\today.

\vspace*{0.5cm}Thành phần đánh giá hội đồng luận văn thạc sĩ bao gồm: \\
1.  (Chủ tịch)\\
2.  (Thư ký)	\\
3.  (Phản biện 1)\\
4.  (Phản biện 2)\\
5.  (Uỷ viên)\\
Xác nhận của Chủ tịch Hội đồng đánh giá luận văn và Trưởng khoa quản lý chuyên ngành sau khi luận văn đã được sửa chữa (nếu có).\\
\begin{tabular}{ p{7cm}p{9cm}}
 \hspace{1cm} CHỦ TỊCH HỘI ĐỒNG & TRƯỞNG KHOA KH$\&$KT Máy Tính\\
 \hspace{1cm}  (Họ tên và chữ ký) & (Họ tên và chữ ký)\\
\end{tabular}\\
\pagebreak
%----------------------------------------------------------------------------------------
\begin{tabular}{ p{7cm}p{9cm}}
 ĐẠI HỌC QUỐC GIA TP. HỒ CHÍ MINH & \textbf{CỘNG HÒA XÃ HỘI CHỦ NGHĨA VIỆT NAM}\\
 \hspace{0.5cm}TRƯỜNG ĐẠI HỌC BÁCH KHOA & \hspace{2cm}\textbf{Độc lập - Tự do - Hạnh phúc}\\
\end{tabular}\\
\vspace*{1cm}
\begin{center}{\fontsize{16pt}{1}\selectfont \textbf{NHIỆM VỤ LUẬN VĂN THẠC SĨ}}\\\end{center}
\begin{tabular}{ ll}
\vspace*{0.1cm}Họ tên học viên: Lê Thị Minh Thùy&\vspace*{0.1cm} MSHV: 13070269\\
\vspace*{0.1cm} Ngày, tháng, năm sinh: 22/01/1986 &\vspace*{0.1cm} Nơi sinh: Đồng Nai\\
\vspace*{0.1cm} Ngành: Khoa học máy tính & \vspace*{0.1cm} Mã số: 60.48.01\\
\multicolumn{2}{l}{\vspace*{1cm}I. TÊN ĐỀ TÀI: Dự đoán xác suất xe buýt về trạm đúng giờ} \\
\vspace*{0.1cm} II.  NHIỆM VỤ VÀ NỘI DUNG: \\
\multicolumn{2}{l}{\vspace*{0.1cm}1. Tìm thuật toán đơn giản, phù hợp để giải bài toán} \\
\multicolumn{2}{l}{\vspace*{3cm}2. Hiện thực để chứng minh thuật toán đã chọn lập trình được} \\

\vspace*{0.1cm} III.  NGÀY GIAO NHIỆM VỤ: ngày 20 tháng 6 năm 2016& \\
\vspace*{0.1cm} IV.  NGÀY HOÀN THÀNH NHIỆM VỤ: \today & \\
\vspace*{0.1cm}V. CÁN BỘ HƯỚNG DẪN: TS. Huỳnh Tường Nguyên & \\
\end{tabular}\\
\vspace*{1cm}
\hfill Tp. HCM, \today. 
\begin{tabular}{ p{7cm}p{9cm}}
 \hspace{1cm} CÁN BỘ HƯỚNG DẪN & TRƯỞNG KHOA KH$\&$KT Máy Tính\\
 \hspace{1cm}  (Họ tên và chữ ký) & (Họ tên và chữ ký)\\
\end{tabular}\\
%----------------------------------------------------------------------------------------
\pagebreak
\begin{center}{\fontsize{16pt}{1}\selectfont \textbf{LỜI NÓI ĐẦU}}\\\end{center}
Để hoàn thành được Luận Văn Thạc Sĩ ngày hôm nay, tôi phải cảm ơn rất nhiều người nhưng trong khuôn khổ giới hạn không được phép trình bày hết, cho nên tôi xin được gửi lời cảm ơn ngắn gọn đến tất cả mọi người. Tôi chân thành cảm ơn quá trình 4 năm học tập chương trình Thạc Sĩ Khoa Học Máy Tính tại trường Đại Học Bách Khoa, có cơ hội trao đổi kinh nghiệm quý báu với thầy cô và các bạn cùng khóa.\\
Trong chương Cơ sở lý thuyết của Luận Văn Thạc Sĩ của mình, tôi có sao chép một phần nội dung trong sách "Thống kê Công nghiệp hiện đại với ứng dụng viết trên R, MINITAB và JMP" của thầy giáo Nguyễn Văn Minh Mẫn, cuốn sách thuộc bản quyền tiếng Việt của Viện Nghiên cứu cao cấp về Toán. Tôi xin phép không trả tiền bản quyền sử dụng, do Luận Văn Thạc Sĩ của tôi chỉ sử dụng cho ngày bảo vệ hội đồng Thạc Sĩ và sau đó bỏ đi. Tôi cam đoan nếu sử dụng chỉ một phần nội dung cuốn sách này cho mục đích thương mại hay bất kỳ hoạt động khoa học nào như đăng báo hay thuyết trình bất cứ đâu, tôi sẽ trả tiền bản quyền.\\
\begin{flushright}
\today\\
Lê Thị Minh Thùy
\end{flushright}
%-----------------------------------------------------------------------------------------
\pagebreak
\begin{center}{\fontsize{16pt}{1}\selectfont \textbf{TÓM TẮT LUẬN VĂN}}\\\end{center}
Dùng xác suất Bayes là lý thuyết nền tảng để dự đoán xác suất xe buýt về trạm đúng giờ với dữ liệu mẫu ngẫu nhiên và số lượng giới hạn.
%-----------------------------------------------------------------------------------------
\pagebreak
\begin{center}{\fontsize{16pt}{1}\selectfont \textbf{ABSTRACT}}\\\end{center}
Bayes' theorem is used to predict the probability of a bus on the specific route arriving the destination on time with the random and limited quantity of sample data.  
%-------------------------------------------------------------------------------------------
\pagebreak
\begin{center}{\fontsize{16pt}{1}\selectfont \textbf{LỜI CAM ĐOAN}}\\\end{center}
Tôi cam đoan rằng, ngoại trừ các kết quả tham khảo từ các công trình khác như đã ghi rõ trong luận văn, các công việc trình bày trong luận văn này do chính tôi thực hiện và không nội dung nào của luận văn này đã được nộp để lấy một bằng cấp ở trường này hoặc trường khác.\\
\begin{flushright}
\today\\
Lê Thị Minh Thùy
\end{flushright}
\iffalse
%comment
\fi
\tableofcontents
\pagebreak
\pagenumbering{arabic}
\chapter{GIỚI THIỆU ĐỀ TÀI} %%%%%%%%%%%%%%%%%%%%%%%%%%%%%%%%
\section{Giới thiệu đề tài}
Theo nghiệp vụ xe buýt, các bác tài xế chỉ có khoảng thời gian cố định cộng thêm linh động trễ thêm vài phút để hoàn tất một lộ trình đã vạch sẵn. Khi lái xe lâu năm trên một lộ trình giống nhau, các bác tài xế khi đi được một phần đoạn đường, họ sẽ ước lượng quãng thời gian còn lại có hoàn thành kịp tiến độ cho quãng đường còn lại hay không.\\
Mục đích Luận Văn: dùng Thống Kê định lượng kinh nghiệm nhắm chừng này.\\

\section{Động cơ}
Hiện nay, có nhiều ứng dụng hay nghiên cứu khoa học khai thác thông tin GPS của các chuyến xe buýt di chuyển trên địa bàn TP. Hồ Chí Minh để rút trích thông tin có ích với mong muốn với tri thức mới tìm được trong dữ liệu thô có thể giúp ích và cải thiện dịch vụ phục vụ của phương tiện giao thông công cộng này.\\
\section{Mục tiêu}
Đề bài Luận Văn cụ thể: Dùng Thống Kê định lượng kinh nghiệm di chuyển trên một lộ trình quen thuộc, dự đoán xe buýt về trạm đích đúng giờ hay trễ giờ.\\
Tác giả không đặt mục tiêu tham vọng giải bài toán này có thể đem ra ứng dụng thực tế được vì bản thân tác giả không phải là nhà thống kê, không có nhiều kinh nghiệm làm việc với kích thước mẫu rất nhỏ có thể đưa ra dự đoán đúng trên kích thước quần thể. Nhưng thông qua Luận Văn, tác giả chứng minh được sử dụng kiến thức Thống Kê cơ bản (không chứng minh công thức Toán học vì tác giả không phải là nhà Toán học), giải được bài toán trên.\\
Ngoài ra, tác giả muốn nhấn mạnh về kết quả dự đoán chỉ mang tính xác suất. Ví dụ, sau khi đi được 2/3 chặng, bác tài xế nhận được kết quả xác suất 99\% xe về trạm đúng giờ. Nhưng gặp sự cố tại 1/3 chặng đường cuối, tình trạng kẹt xe nặng, xe về trạm đích không đúng giờ, trái ngược hoàn toàn với con số xác suất đúng giờ "rất đẹp" 99\%. Bác tài xế phải chấp nhận xác suất 1\% về trạm trễ giờ vẫn có khả năng xảy ra lớn do bác tài xế không thể điều khiển tình trạng xe chạy tại 1/3 chặng đường cuối. Nếu không đặt nặng việc dự báo bắt buộc phải chính xác, các con số xác suất được thông báo vẫn có ý nghĩa chừng mực nếu bác tài xế biết rằng các con số xác suất này sử dụng những chuyến di chuyển trong quá khứ để dự báo tương lai cho chuyến đi hiện tại.\\
Luận Văn nhấn mạnh việc hiểu được định nghĩa xác suất, hiểu được các sự kiện xảy ra trên thực tế không bao giờ được dự đoán chắc chắn 100\% xảy ra, mà chỉ dự đoán khả năng có thể xảy ra. Đến đây, tác giả mong đợi người đọc hiểu được ý nghĩa ứng dụng thực tế của Luận Văn.\\
Tác giả nhắc lại đề tài Luận Văn: Dùng Thống Kê định lượng kinh nghiệm di chuyển trên một lộ trình quen thuộc, dự đoán xác suất xe buýt về trạm đích đúng giờ.
\section{Phương pháp nghiên cứu}
Phương pháp nghiên cứu là thu thập số lượng dữ liệu mẫu di chuyển giới hạn, quan sát các bước di chuyển được ghi nhận cách nhau 20 giây trên 2/3 lộ trình đi từ BX Củ Chi - BX An Sương, trực quan hóa dữ liệu này, cộng thêm kiến thức lập luận thực tế ví dụ như nếu chuyến đi đó có nhiều bước di chuyển dài trong từng khoảng 20 giây thì chuyến đi đó có khả năng về trạm đúng giờ và chấp nhận xác suất, cho biết khả năng xảy ra, chấp nhận có yếu tố ngẫu nhiên xảy ra ảnh hưởng đến kết quả dự đoán. Quan sát dữ liệu cho ta khởi đầu cách tiếp cận, lấy cảm hứng kiến thức xác suất và thống kê để tìm ra phương pháp giải bài toán.  
\section{Một số kết quả thu được}
Tác giả nói qua sơ lược phương pháp giải bài toán, có thể lúc này người đọc chưa hiểu hết.\\
\textbf{\underline{Bước 1}}: Tuyến xe buýt 74 ngẫu nhiên được chọn và chọn lộ trình cố định BX Củ Chi - BX An Sương. Không sử dụng sức mạnh tính toán hiệu năng cao trên kích thước dữ liệu siêu lớn. Chọn mẫu dữ liệu có số lượng giới hạn và ngẫu nhiên.\\
\textbf{\underline{Bước 2}}: Làm sạch dữ liệu
\begin{enumerate}
\item Loại bỏ những chuyến đi về trạm đích trễ một cách bất thường so với thông thường. Nghĩa là thông thường chỉ mất 45 phút để hoàn thành lộ trình, cần loại bỏ những chuyến về trạm đích quá trễ, ví dụ sau 1 tiếng, vì tìm hiểu chúng chẳng có ích gì cả, chúng xảy ra vì hiện tượng hiếm xảy nó và chúng chẳng đại diện cho số đông. Thống kê tìm những đặc điểm đúng cho số đông. 
\item Do giới hạn về kỹ thuật thu nhận dữ liệu GPS sớm hơn hay trễ hơn, chuẩn hóa những bước di chuyển trong từng khoảng 20 giây.
\end{enumerate}
\textbf{\underline{Bước 3}} Làm việc với dữ liệu, trong một chuyến đi, cắt bỏ 1/3 đoạn đường sau, chỉ giữ lại 2/3 đoạn đường trước, dán nhãn "đúng giờ" hay "trễ giờ" nhờ thời gian hoàn thành chuyến đi phân phối chuẩn. Trực quan những bước di chuyển của mỗi chuyến đi.\\
Quan sát và nhận thấy trên thực tế không tồn tại hai chuyến xe cùng lộ trình có những bước di chuyển cách nhau 20 giây giống nhau hoàn toàn cho nên đừng đi tìm một công thức chuẩn cho một chuyến xe về trạm đúng giờ là vô ích, mà hãy quan sát những bước nhảy trong khoảng thời gian đều 20 giây mà nhận ra được nếu chuyến đi đó có nhiều bước di chuyển dài trong từng khoảng 20 giây thì chuyến đi đó có khả năng về trạm đúng giờ.\\
Phân chia dữ liệu mẫu thành 2 phần: phần để học có kèm theo kết quả về trạm đích, phần để kiểm tra đã loại bỏ nhãn đúng giờ hay trễ giờ tại trạm đích.\\
\textbf{\underline{Bước 4}} Phân loại những bước di chuyển: bước đi rất nhỏ 0-15 (km/h), bước đi nhỏ 15-30 (km/h), bước đi trung bình 30-45 (km/h), bước đi xa 45-60 (km/h), bước đi rất xa trên 60 km/h.\\
\textbf{\underline{Bước 5}} Không đếm số lần xảy ra những bước chuyển rời rạc mà sử dụng Kernel Density Estimation dự đoán hàm mật độ xác suất, ký hiệu pdf (probability density function) của biến ngẫu nhiên.\\ 
\textbf{\underline{Bước 6}} Sử dụng công thức xác suất Bayes nền tảng $\mathbb{P}(B|A) = \frac{\mathbb{P}(B) \cdot \mathbb{P}(A|B)}{\mathbb{P}(A)}$ để giải bài toán dán nhãn. Công thức xác suất Bayes phù hợp để dự đoán vì khi ta quan sát biến cố A xảy ra và có xác suất có điều kiện $\mathbb{P}(A|B)$ được biết trước, lý thuyết Bayes cho ta cách tính để đánh giá xác suất một sự kiện B chưa quan sát được xảy ra sau khi biến cố A đã xảy ra.\\
Tác giả nhắc lại đề tài Luận Văn: Dùng Thống Kê định lượng kinh nghiệm di chuyển trên một lộ trình quen thuộc, dự đoán xác suất xe buýt về trạm đích đúng giờ.
\pagebreak
\chapter{CƠ SỞ LÝ THUYẾT}
\section{Lý thuyết xác suất cơ bản}
\subsection*{Tổng thể và mẫu}
Kham khảo trang 19, 24, 25 chương 2 từ sách \cite{TKCNUDR}
\begin{itemize}
\item Một \textbf{tổng thể} (còn được gọi là quần thể) thống kê là một tập các phần tử có một thuộc tính chung nhất định. 
\\Ví dụ, tập hợp tất cả các chuyến xe buýt từ BX Củ Chi - BX An Sương của tuyến 74 vào ngày 17 tháng 10 năm 2016 là một tổng thể hữu hạn và có thực.\\
Một ví dụ khác, tập tất cả các chuyến xe buýt từ BX Củ Chi - BX An Sương của tuyến 74 có thể về trạm trễ trong điều kiện: mưa nhiều, đường ngập, hẹp, rào chắn lô cốt, quá tải xe máy xuyên suốt cả ngày và va chạm xe máy thường xuyên. Tổng thể này là vô hạn và giả định.
\item Một \textbf{mẫu} là một tập hợp các phần tử trong một tổng thể nhất định. Một mẫu thường được chọn ra từ một tổng thể với mục tiêu quan sát các đặc tính của tổng thể ấy và đưa ra các quyết định thống kê có liên quan đến các đặc trưng tương ứng. \\
Chẳng hạn, xét vài trăm triệu chuyến xe từ BX Củ Chi - BX An Sương của tuyến 74, công ty quản lý muốn tìm ra được đặc trưng tình trạng di chuyển 2/3 chặng đường như thế nào để cảnh báo sớm tình trạng đến đích trễ cho các bác tài xế. Nếu không sử dụng sức mạnh tính toán của điện toán đám mây, chỉ có giới hạn sức tính toán của con người, những nhà thống kê sẽ rút ra đặc trưng của vài trăm triệu dữ liệu bằng cách làm việc trên mẫu ngẫu nhiên rút ra từ vài trăm triệu dữ liệu trên. Những thủ tục lấy mẫu như vậy để đưa ra các quyết định thống kê được gọi là phương pháp lấy mẫu chấp nhận. Ngoài ra, Toán Thống Kê cung cấp phương pháp ước lượng bằng cách sử dụng các mẫu chọn ra từ các tổng thể hữu hạn, bao gồm cả việc lấy mẫu ngẫu nhiên có hoàn lại và lấy mẫu ngẫu nhiên không hoàn lại.\\
\end{itemize}
Như vậy, trong Toán Thống Kê, tồn tại các phương pháp như phương pháp thí nghiệm thống kê, phương pháp lấy mẫu chấp nhận, phương pháp kiểm định giả thuyết,.. để từ mẫu ngẫu nhiên đủ rút ra đặc trưng của tổng thể với xác suất xảy ra cao. Nhưng để thực hiện được điều này, nó vượt quá kiến thức kỹ sư máy tính. Cho nên Luận Văn này không đặt tham vọng, giải trên mẫu có thể kết luận trên tổng thể với xác suất xảy ra cao.\\
Ngoài ra, tôi có đưa thêm một số giả định trong khi xem xét mẫu được lấy ngẫu nhiên trong Luận Văn: 
\begin{itemize}
\item Tôi tập trung vào đo lường chuyến đi trong một khoảng thời gian cụ thể, tháng 9 năm 2016, các mẫu của tôi không rải rác qua các tháng, các năm. 
\item  Chúng ta chỉ dự đoán được một phần trong nhiều hiện tượng mà ta gặp phải. Xem xét tất cả các chuyến xe trên mọi điều kiện không thể kiểm soát được trên thực tế là quá tốn kém và không thực tế. Cho nên các yếu tố bên ngoài như thời điểm xuất phát, thời tiết, điều kiện mặt đường,... được coi là các yếu tố độc lập để giảm sự phức tạp giải bài toán.
\end{itemize}  
\subsection*{Biến cố và không gian mẫu: Diễn tả hình thức của thí nghiệm}
Kham khảo trang 52, 55, 59, 60, 61, 63 chương 3 từ sách \cite{TKCNUDR}\\
Ta thấy cùng một tuyến đường xe buýt 74 cố định, từ BX Củ Chi - BX An Sương, các chuyến xe trong cùng tháng 9 có thời gian hoàn thành khác nhau, không biết trước được một cách chắc chắn. Nguyên nhân là do những yếu tố bên ngoài như mặt đường, thời tiết, giờ cao điểm, thấp điểm,..đều có ảnh hưởng đến kết quả. Toán Thống Kê giúp ta có phương pháp làm việc trên dữ  liệu có tính ngẫu nhiên như vậy.\\
\textbf{Không gian mẫu} là tập hợp của tất cả các kết cục có thể của một thí nghiệm cụ thể. Chẳng hạn, thí nghiệm tung một đồng xu, kết quả ngẫu nhiên là ngửa (Head, $\mathcal{H}$) hoặc sấp (Tail, $\mathcal{T}$), cho ta không gian mẫu $\mathcal{S= \{H,T}\}$. Các \textbf{biến cố sơ cấp} hay \textbf{điểm mẫu} là những phần tử của $\mathcal{S}$.\\
\subsection*{Xác suất của biến cố}
Thông thường chúng ta gộp tất cả các biến cố vào một tập $\mathcal{Q}$ := $\{ A: A \subset \mathcal{S}$ là một biến cố$\}$, gọi là tập các biến cố.\\
Xét một hàm $\mathbb{P}: \mathcal{Q} \rightarrow \mathbb{R}$ xác định trên $\mathcal{Q}$, gán cho mỗi biến cố A một số thực, ký hiệu A $\mapsto$ $\mathbb{P}$[A], $\mathbb{P}$[A] (hay $\mathbb{P}$(A)) là khả năng hoặc cơ hội mà biến cố A xảy ra. $\mathbb{P}$ được gọi là hàm xác suất khi thỏa mãn những tiên đề cơ bản sau đây:\\
\begin{itemize}
\item $\textbf{A1}$ Xác suất là không âm, $\mathbb{P}$(A) $\geq$ 0
\item $\textbf{A2}$ Không gian mẫu $\mathcal{S}$ có xác suất 1, $\mathbb{P}(\mathcal{S})$=1
\item $\textbf{A3}$ Xác suất của các biến cố rời nhau. Khi có hai biến cố A, B mà A $\cap$ B = $\emptyset$ thì\\
\begin{center}
$ \mathbb{P}(A \cup B) = \mathbb{P}(A $ hay $B) = \mathbb{P}(A) + \mathbb{P}(B)$
\end{center}
Tổng quát hơn, nếu ta có $E_{1}, E_{2},...,E_{n}$ (n $\geq$ 1) là các biến cố rời nhau từng đôi một thì\\
\[
\mathbb{P}\left[\bigcup_{i=1}^n E_{i}\right]=\sum_{i=1}^n \mathbb{P}[E_i]
\]
\end{itemize}
\subsection*{Xác suất có điều kiện và sự độc lập của các biến cố}
Khi các biến cố khác nhau có liên quan, việc thực hiện một biến cố có thể cung cấp cho ta thông tin liên quan để cải thiện, nâng cao khả năng đánh giá của ta về các biến cố khác. Biến cố B đã xảy ra, tức là $\mathbb{P}[B]$ > 0, xác suất biến cố A cũng xảy ra là gì? Suy nghĩ một cách tích cực, ta thấy nên thu hẹp không gian mẫu S tới không gian trong đó B đã xảy ra (nhằm mục đích so sánh giữa A $\cap$ B và B)
\textbf{Xác suất có điều kiện} của biến cố A cho biết biến cố B đã xảy ra, $\mathbb{P}[B]$ > 0 là
\[
\mathbb{P}(A|B)=\frac{\mathbb{P}(AB)}{\mathbb{P}(B)}=\frac{\mathbb{P}(A\cap B)}{\mathbb{P}(B)}
\]
Theo đó xác suất đồng thời của hai biến cố A và B là
\[
\mathbb{P}(AB)=\mathbb{P}(A\cap B)=\mathbb{P}(B)\cdot \mathbb{P}(A|B)
\]
Ví dụ: Thí nghiệm đo chiều dài của một thanh thép. Không gian mẫu là $\mathcal{S}$ = (19.5, 20.5)[cm]. Hàm xác suất gán bất kỳ một khoảng tập con của S một xác suất bằng chiều dài của nó. Cho A = (19.5,20.1), nghĩa là $\mathcal{P}(A)$ = 20.1-19.6=0.6 và B = (19.8,20.5), nghĩa là $\mathcal{P}(B)$ = 20.5-19.8=0.7. Khoảng tập con chung giữa A$\cap$B=(19.8,20.1), nghĩa là $\mathcal{P}(A\cap B)$ = 20.1-19.8=0.3.\\
Cho một độ dài và biết thêm điều kiện độ dài này thuộc khoảng B, và chúng ta phải đoán xem nó có thuộc về A không? Ta tính xác suất có điều kiện
\[
\mathbb{P}(A|B)=\frac{\mathbb{P}(AB)}{\mathbb{P}(B)}=\frac{\mathbb{P}(A\cap B)}{\mathbb{P}(B)}=\frac{0.3}{0.7}=0.4286
\]
Mặt khác, nếu thông tin độ dài thuộc về B không biết trước, thì với độ dài của câu hỏi trên, xác suất nó thuộc về A bằng $\mathbb{P}(A)$=0.6. Vậy có một sự khác biệt giữa xác suất có điều kiện và không điều kiện. Điểu này cho thấy hai biến cố A và B là phụ thuộc.\\
Hai biến cố A và B được gọi là \textbf{độc lập} nếu
\[
\mathbb{P}(A|B)=\mathbb{P}(A)
\]
Nói khác đi, biến cố A và B độc lập nếu sự xuất hiện của A không có liên quan theo bất kỳ cách nào đến sự xuất hiện của B: $\mathbb{P}(A|B)=\mathbb{P}(A)$ và $\mathbb{P}(B|A)=\mathbb{P}(B)$\\
Nếu A và B là các biến cố độc lập thì 
\[
\mathbb{P}(A)=\mathbb{P}(A|B)=\frac{\mathbb{P}(A\cap B)}{\mathbb{P}(B)}
\]
hay tương đương với 
\[
\mathbb{P}(AB)=\mathbb{P}(A\cap B)=\mathbb{P}(A)\cdot\mathbb{P}(B)
\]
nghĩa là xác suất hai biến cố độc lâp đồng thời xảy ra bằng tích các xác suất riêng lẻ.\\
Tổng quát, các biến cố $A_{1}, A_{2},..., A_{n}$ là n biến cố \textbf{độc lập lẫn nhau} thì
\[
\mathbb{P}\left[\bigcap_{i=1}^n A_j \right] = \prod_{i=1}^n \mathbb{P}[A_i]
\]
\subsection*{Biến ngẫu nhiên}
Kham khảo trang 62, 63 chương 3 từ sách \cite{TKCNUDR}\\
\textbf{Định nghĩa biến ngẫu nhiên}: Một biến ngẫu nhiên là một hàm giá trị thực $X(\omega)$ (hay X) xác định trên một không gian mẫu $\mathcal{S}$, sao cho các biến cố $\{\omega \in \mathcal{S}: X(\omega) \leq x\}$ có thể được gán các xác suất, với mọi  $-\infty < x < \infty$. Ta ghi $X: \mathcal{S} \rightarrow \mathbb{R}$. Thật vậy, với bất kỳ $x \in \mathbb{R}$, tập tiền ảnh $\{\omega \in \mathcal{S}:X(\omega) \leq x\}\subseteq \mathcal{S}$ rõ ràng là một biến cố, và được ký hiệu là A = X $\leq$ x hay $\{X \leq x\}$. Vậy xác suất $\mathbb{P}[A]$=$\mathbb{P}[X \leq x]$ luôn tồn tại.\\
Kham khảo ví dụ trang 20 chương 2 từ \cite{TKCNUDR}\\
Xét một thí nghiệm trong đó ta tung một đồng xu một lần. Giả sử đồng xu là cân đối và đồng chất để khả năng xuất hiện một trong hai mặt là như nhau. Hơn nữa, giả định rằng hai mặt của đồng xu được gán nhãn bởi "0" và "1". Nói chung, chúng ta không thể dự đoán chắc mặt nào sẽ hiện ra. Nếu mặt "0" xuất hiện thì chúng ta gán cho một biến X giá trị 0; nếu mặt "1" xuất hiện, ta gán cho X giá trị 1. Vì những giá trị mà X sẽ nhận được trong một chuỗi các thử nghiệm như vậy là không thể dự đoán được một cách chắc chắn, nên chúng ta gọi là X một biến ngẫu nhiên. Một ví dụ cụ thể cho chuỗi ngẫu nhiên gồm các giá trị 0, 1 được tạo ra bằng cách này là như sau: 0,1,1,0,1,0,1,1,1,1,1,1,0,1,1,1,1\\
\subsection*{Biến ngẫu nhiên rời rạc}
\textbf{Định nghĩa biến ngẫu nhiên rời rạc} X(.) là biến có một phạm vi $\mathcal{S}_X = X(\mathcal{S})$ là tập giá trị rời rạc (hữu hạn hoặc vô hạn đếm được, nghĩa là có lượng số không quá lượng số tập tự nhiên $\mathbb{N}$)\\
Trường hợp hữu hạn phần tử thì ta thường ghi tập giá trị\\
\[
\mathcal{S}_X = \{x_0, x_1, x_2, ..., x_{m-1}, x_m\}, m \in \mathbb{N}
\]
Trường hợp X(.) có phạm vi vô hạn đếm được thì ta ghi\\
\[
\mathcal{S}_X = \{x_0, x_1, x_2, ..., x_{m-1}, x_m, ...\}, 
\]
tập này có cùng lượng số với tập $\mathbb{N}$.\\
Đối với một biến ngẫu nhiên X rời rạc, ta có các khái niệm sau\\
\textbf{Hàm mật độ xác suất} là\\
\[
p(x) = \mathbb{P}[X=x]= \mathbb{P}[\{\omega: X(\omega)=x\}], x \in \mathcal{S}_X
\]
$p(x)$ là xác suất mà X nhận một giá trị cụ thể x $\in$ $\mathcal{S}_X$. Ta phải có\\
\begin{center}
$p(x) \geq 0$ và $\sum_{x \in \mathcal{S}_X} p(x) = 1$
\end{center}
\textbf{Phân phối xác suất} của một biến ngẫu nhiên mô tả cách các xác suất được phân phối trên các giá trị của biến ấy. Tập giá trị $\mathcal{S}_X$ = $\{x_0,x_1,x_2,...,x_{m-1}, x_{m}\}$(còn được gọi không gian mẫu của X là hữu hạn), cho ta \textbf{bảng phân phối xác suất} của X, được cho bởi\\
\begin{center}
\begin{tabular}{ rccccc }
\specialrule{.1em}{.05em}{.05em} 
X & $x_0$ & $x_1$ & ... & $x_{m-1}$ & $x_{m}$\\
\hline
p$_k$:=$p(x_k$)=$\mathbb{P}$[X=$x_k$] & $p_0$ & $p_1$ & ... & $p_{m-1}$ & $p_m$\\
\specialrule{.1em}{.05em}{.05em} 
\end{tabular}
\end{center}
\textbf{Hàm tích lũy xác suất} của X được tính bằng cách lấy tổng các xác suất của các giá trị $x_k \in \mathcal{S}_X$ mà $x_k \leq x$, đó là 
\[
F_{X}(x) = \mathbb{P}[X\leq x] = \mathbb{P}[{\omega \in S: X(\omega) \leq x}]=\sum_{x_k\leq x} p(x_k)
 = \sum_{x_k\leq x} p_k,   x \in \mathbb{R}
\] 
\textbf{Tham số đặc trưng của biến rời rạc} X với hàm mật độ $p_k = p(x_k)$, tập giá trị $\mathcal{S}_X$:
\begin{itemize}
\item Kỳ vọng (hay trung bình) $\mu$ và phương sai V[X] = $\sigma^2$ của X cho bởi
\[
\mu = \mathbb{E}[X] = \sum_{x_k \in \mathcal{S}_X} x_k p_k
\]
\[
\sigma^2 = \mathbb{E}[(X-\mu)^2] = \sum_{x_k \in \mathcal{S}_X} [x_k - \mu]^2 p_k
\]
\item Độ lệch chuẩn của biến X là $\sigma_{X} = \sigma = \sqrt{Var(X)}$
\end{itemize}
\subsection*{Biến ngẫu nhiên liên tục}   
\textbf{Định nghĩa biến ngẫu nhiên liên tục} X(.) là liên tục khi nó có phạm vi bao gồm khoảng con (hay toàn bộ) tập số thực, nghĩa là $\mathcal{S}_X \in \mathbb{R}$. Một cách toán học thì biến X liên tục thì 
\begin{itemize}
\item X nhận vô hạn giá trị không đếm được, $\mathcal{S}_X \subset \mathbb{R}$ và 
\item tồn tại một hàm số không âm $f(u)$ thỏa
\[
F(x) = \mathbb{P}(X \leq x) = \int_{-\infty}^{x} f(u)d(u), \infty < x < -\infty 
\]
F($x$) gọi là hàm phân phối (tích lũy) xác suất (c.d.f) của X, f(u) là hàm mật độ của X\\
Tính chất của hàm phân phối xác suất F gồm:
\begin{itemize}
\item F liên tục, và
\item $\lim_{x->-\infty}F(x)=0$, $\lim_{x->\infty}F(x)=1$
\item F không giảm, nghĩa là nếu $x_1$ < $x_2$ thì F($x_1$) $\leq$ F($x_2$), và
\item Quan hệ với f: hàm phân phối xác suất F($x$) có đạo hàm $\frac{dF(x)}{dx}$ = $f(x)$   
\end{itemize}
\end{itemize}
Tính chất của hàm mật độ xác suất f gồm:
\begin{itemize}
\item f(u) $\geq$ 0, $\forall$u, hay đường biểu diễn f nằm trên phần dương. Hơn thế $\int_{-\infty}^{\infty} f(u)du$ = 1 = F($x$)
\item Xác suất của biến cố "a $\leq$ X $\leq$ b" là diện tích giới hạn bởi hàm mật độ, trục hoành và 2 đường thẳng u = a và u = b:
\[
\mathbb{P}(a \leq X \leq b) = \mathbb{P}(a \le X \le b) = \mathbb{P}(a \leq X \le b) = \int_{a}^{b} f(u)du = F(b) - F(a)
\] 
hay là 
\[
\mathbb{P}(X \geq b) = \int_{b}^{\infty} f(u)du = 1 - F(b)
\]
\item Đạo hàm $f(x) = \frac{dF(x)}{dx}$ có thể không tồn tại ở một số hữu hạn giá trị x, trong khoảng hữu hạn bất kỳ.   
\end{itemize}
\subsection*{Công thức Bayes và ứng dụng của nó}
Công thức Bayes cho ta cách thức cơ bản để đánh giá bằng chứng trong các dữ liệu liên quan đến các thông số chưa biết, hoặc một số biến cố không quan sát được. Giả sử rằng một thí nghiệm ngẫu nhiên cho kết quả trong một biến cố A (hoặc phần bù của nó), ngoài ra các kết quả này phụ thuộc vào một biến cố B mà ta không trực tiếp quan sát được, nhưng xác suất có điều kiện $\mathbb{P}$(A|B) là được biết trước.\\
Bayes nói biến cố quan sát được A có liên quan đến biến cố B không quan sát được thông qua các xác suất có điều kiện. Để cân nhắc bằng chứng cho thấy A có ảnh hưởng trên B, diễn đạt bởi $\mathbb{P}(B|A)$ đầu tiên chúng ta giả định một xác suất $\mathbb{P}$(B) mà được gọi là xác suất tiên nghiệm. Xác suất tiên nghiệm $\mathbb{P}(B)$ thể hiện mức độ tin tưởng của chúng ta vào sự xảy ra của biến cố B. Ta luôn có điều kiện sau, cho phép tính xác suất hậu nghiệm $\mathbb{P}(B|A)$, là xác suất B xảy ra sau khi quan sát A (nên có mẫu số $\mathbb{P}$(A))\\
\[
\mathbb{P}(B|A) = \frac{\mathbb{P}(B) \cdot \mathbb{P}(A|B)}{\mathbb{P}(A)}
\]
Tổng quát, giả sử $\{$B$_1$,...,B$_m\}$ là một phân vùng của không gian mẫu. Các biến cố B$_1$,...,B$_m$ không trực tiếp quan sát hay kiểm chứng được, nhưng các xác suất có điều kiện $\mathbb{P}(A|B_i)$ là được biết trước. Giả định các xác suất tiên nghiêm là $\mathbb{P}(B_i)$, thể hiện mức độ tin tưởng của ta vào sự xảy ra của biến cố B$_i$. Sau khi quan sát A ta chuyển đổi các xác suất tiên nghiệm của B$_{i}$ thành các xác suất hậu nghiệm $\mathbb{P}(B_i|A)$, theo trên\\
\[
\mathbb{P}(B_i|A) = \frac{\mathbb{P}(B_i) \cdot \mathbb{P}(A|B_i)}{\mathbb{P}(A)}
\]
Vì $\{$B$_1$,...,B$_m\}$là một phân vùng 	của $\mathcal{S}$, $\mathcal{S}=\bigcup_{j=1}^m $B$_j$, nên A = A$\mathcal{S}$ = $\bigcup_{j=1}^m$ AB$_j$, vậy\\
\[
\mathbb{P}(A)=\sum_{j=1}^m \mathbb{P}(AB_j) = \sum_{j=1}^m \mathbb{P}(B_j)\cdot \mathbb{P}(A|B_j)
\]
Công thức Bayes tổng quát là:\\
\[ %\mathlarger{
  \mathbb{P}(B_{i}|A)= \dfrac{\mathbb{P}(B_{i})\cdot \mathbb{P}(A|B_{i})}{\mathbb{P}(A)}
            =\dfrac{\mathbb{P}(B_{i})\cdot \mathbb{P}(A|B_{i})}{\sum_{j=1}^{m} \mathbb{P}(B_j) \cdot \mathbb{P}(A|B_j)}
%}
\]
\subsection*{Ví dụ phân loại dựa vào xác suất Bayes}
\begin{center}
\begin{tabular}{ cccccc }
\specialrule{.2em}{.1em}{.1em} 
RID & age & income & student & credit\_rating & Class: buys\_computer\\
\specialrule{.2em}{.1em}{.1em} 
1 & youth &	high	 & no & 	fair & no\\
2 & 	youth &	high	 & no & 	excellent & 	no\\
3 & middle\_aged & high & no & fair & yes\\
4 & 	senior & medium & no & fair & yes\\
5 & 	senior & low & yes & fair & yes\\
6 & 	senior & low	 & yes & excellent & no\\
7 & 	middle\_aged & low & 	yes & excellent & yes\\
8 & 	youth & 	medium & no & fair & no\\
9 & 	youth & 	low & yes & fair & yes\\
10 & senior & medium & yes & fair & yes\\
11 & youth & medium & yes & excellent & yes\\
12 & middle\_aged & medium & no & excellent & yes\\
13 & middle\_aged	 & high & yes & fair & yes\\
14 & senior & medium	 & no & 	excellent & 	no\\
\specialrule{.2em}{.1em}{.1em}
\end{tabular}\\
\captionof{table}{Bảng dữ liệu} 
\end{center}
\begin{table}[ht]
\begin{tabular}{ |c|c|c| }
\specialrule{.1em}{.05em}{.05em} 
\multirow{2}{*}{age} & \multicolumn{2}{c|}{buys\_computer} \\
& yes & no \\
\hline
youth & 2 & 3  \\
middle\_aged & 4 & 0 \\
senior & 3 & 2 \\
\specialrule{.1em}{.05em}{.05em} 
\end{tabular}
\hspace{7mm}
\vspace{1mm}
\begin{tabular}{|c|c|c|}
\specialrule{.1em}{.05em}{.05em} 
\multirow{2}{*}{income} & \multicolumn{2}{c|}{buys\_computer} \\
& yes & no \\
\hline
low & 3 & 1  \\
middle & 4 & 2 \\
high & 2 & 2 \\
\specialrule{.1em}{.05em}{.05em} 
\end{tabular}\\
\begin{tabular}{ |c|c|c| }
\specialrule{.1em}{.05em}{.05em} 
\multirow{2}{*}{credit\_rating} & \multicolumn{2}{c|}{buys\_computer} \\
& yes & no \\
\hline
fair & 6 & 2  \\
excellent & 3 & 3 \\
\specialrule{.1em}{.05em}{.05em} 
\end{tabular}
\hspace{7mm}
\vspace{1mm}
\begin{tabular}{ |c|c|c| }
\specialrule{.1em}{.05em}{.05em} 
\multirow{2}{*}{student} & \multicolumn{2}{c|}{buys\_computer} \\
& yes & no \\
\hline
yes & 6 & 1  \\
no & 3 & 4 \\
\specialrule{.1em}{.05em}{.05em} 
\end{tabular}
\captionof{table}{Bảng thông tin tóm tắt (từ bảng 2.1)}
\end{table}
Gọi x$_i$ tương ứng là một dòng dữ liệu ở bảng 2.1\\
Cho dòng dữ liệu\\ 
x$_{15}$ = (age = youth, income = medium, student = yes, credit\_rating = fair)\\ 
Câu hỏi: x$_{15}$ được phân loại vào \textit{buys\_computer = yes} hay \textit{buys\_computer = no}?\\
Từ bảng 2.2, ta tính thêm thông tin xác suất
\[
\begin{aligned}
\mathbb{P}(\text{buys\_computer = yes}) & =  \frac{\sum_{\mathcal{S}=\{\text{buys\_computer = yes}\}} x_i}
{\sum_{\mathcal{S}=\{\text{buys\_computer = yes, buys\_computer = no }\}} x_i} \\
         							& = \frac{9}{14} = 0.643\\
\mathbb{P}(\text{buys\_computer = no}) & =  \frac{\sum_{\mathcal{S}=\{\text{buys\_computer = no}\}} x_i}
{\sum_{\mathcal{S}=\{\text{buys\_computer = yes, buys\_computer = no }\}} x_i} \\
									&=\frac{5}{14} = 0.357\\
\end{aligned}
\]
Sử dụng xác suất có điều kiện, 
\[
\mathbb{P}(A|B)=\frac{\mathbb{P}(A\cap B)}{\mathbb{P}(B)}
\]
Ta tính
\[
\begin{aligned}
\mathbb{P}(\text{age = youth | buys\_computer = yes}) & =  \frac{\sum_{\mathcal{S}=\{\text{age=youth}\}\cap\mathcal{S}=\{\text{buys\_computer = yes}\}} x_i}
{\sum_{\mathcal{S}=\{\text{buys\_computer = yes }\}} x_i} \\
									&= \frac{2}{9} = 0.222\\
\mathbb{P}(\text{age = youth | buys\_computer = no}) & =  \frac{\sum_{\mathcal{S}=\{\text{age=youth}\}\cap\mathcal{S}=\{\text{buys\_computer = no}\}} x_i}
{\sum_{\mathcal{S}=\{\text{buys\_computer = no }\}} x_i} \\
									&= \frac{3}{5} = 0.600\\
\mathbb{P}(\text{income = medium | buys\_computer = yes}) & =  \frac{\sum_{\mathcal{S}=\{\text{income = medium}\}\cap\mathcal{S}=\{\text{buys\_computer = yes}\}} x_i}
{\sum_{\mathcal{S}=\{\text{buys\_computer = yes}\}} x_i} \\
&= \frac{4}{9} = 0.444\\
\mathbb{P}(\text{income = medium | buys\_computer = no}) & =  \frac{\sum_{\mathcal{S}=\{\text{income = medium}\}\cap\mathcal{S}=\{\text{buys\_computer = no}\}} x_i}
{\sum_{\mathcal{S}=\{\text{buys\_computer = no }\}} x_i} \\
&= \frac{2}{5} = 0.400\\
\mathbb{P}(\text{student = yes | buys\_computer = yes}) & =  \frac{\sum_{\mathcal{S}=\{\text{student = yes}\}\cap\mathcal{S}=\{\text{buys\_computer = yes}\}} x_i}
{\sum_{\mathcal{S}=\{\text{buys\_computer = yes}\}} x_i} \\
&= \frac{6}{9} = 0.667\\
\mathbb{P}(\text{student = yes | buys\_computer = no}) & =  \frac{\sum_{\mathcal{S}=\{\text{student = yes}\}\cap\mathcal{S}=\{\text{buys\_computer = no}\}} x_i}
{\sum_{\mathcal{S}=\{\text{buys\_computer = no }\}} x_i} \\
&= \frac{1}{5} = 0.200\\
\mathbb{P}(\text{credit\_rating = fair | buys\_computer = yes}) & =  \frac{\sum_{\mathcal{S}=\{\text{credit\_rating = fair}\}\cap\mathcal{S}=\{\text{buys\_computer = yes}\}} x_i}
{\sum_{\mathcal{S}=\{\text{buys\_computer = yes }\}} x_i} \\
&= \frac{6}{9} = 0.667\\
\mathbb{P}(\text{credit\_rating = fair | buys\_computer = no}) & =  \frac{\sum_{\mathcal{S}=\{\text{credit\_rating = fair}\}\cap\mathcal{S}=\{\text{buys\_computer = no}\}} x_i}
{\sum_{\mathcal{S}=\{\text{buys\_computer = no }\}} x_i} \\
&= \frac{2}{5} = 0.400\\
\end{aligned}
\]
Sử dụng các công thứ sau
\[
\mathbb{P}(AB)=\mathbb{P}(A\cap B)=\mathbb{P}(B)\cdot\mathbb{P}(A|B)
\]
\[
\mathbb{P}\left[\bigcap_{i=1}^n A_j \right]=\prod_{i=1}^n \mathbb{P}(A_j)
\]
Ta tính\\
Do các biến cố age, income, student, credit\_rating là các biến cố độc lập với nhau, cho nên\\
$\mathbb{P}$((age = youth, income = medium, student = yes, credit\_rating = fair)|buys\_computer = yes) \\
= $\mathbb{P}$(age = youth | buys\_computer = yes) $\times$ $\mathbb{P}$(income = medium | buys\_computer = yes)\\
$\times$ $\mathbb{P}$(student = yes | buys\_computer = yes) $\times$ $\mathbb{P}$(credit\_rating = fair | buys\_computer = yes)\\
= 0.222 $\times$ 0.444 $\times$ 0.667 $\times$ 0.667 = 0.044.\\
$\mathbb{P}$((age = youth, income = medium, student = yes, credit\_rating = fair)|buys\_computer = no) \\
= $\mathbb{P}$(age = youth | buys\_computer = no) $\times$ $\mathbb{P}$(income = medium | buys\_computer = no)\\
$\times$ $\mathbb{P}$(student = yes | buys\_computer = no) $\times$ $\mathbb{P}$(credit\_rating = fair | buys\_computer = no)\\
= 0.600 $\times$ 0.400 $\times$ 0.200 $\times$ 0.400 = 0.019.\\
Cuối cùng, ta tính\\
$\mathbb{P}$((age = youth, income = medium, student = yes, credit\_rating = fair, buys\_computer = yes))\\
= $\mathbb{P}$((age = youth, income = medium, student = yes, credit\_rating = fair)|buys\_computer = yes) \\
$\times$ $\mathbb{P}$(buys\_computer = yes)\\
= 0.044 $\times$ 0.643 = 0.028\\
$\mathbb{P}$((age = youth, income = medium, student = yes, credit\_rating = fair, buys\_computer = no)) \\
= $\mathbb{P}$((age = youth, income = medium, student = yes, credit\_rating = fair)|buys\_computer = no) \\
$\times$ $\mathbb{P}$(buys\_computer = no)\\
= 0.019 $\times$ 0.357 = 0.007\\
Ta thấy xác suất biến cố (age = youth, income = medium, student = yes, credit\_rating = fair,buys\_computer = yes) xảy ra cao hơn so với xác suất biến cố (age = youth, income = medium, student = yes, credit\_rating = fair, buys\_computer = no). Do đó x$_{15}$ = (age = youth, income = medium, student = yes, credit\_rating = fair) được phân loại vào \textit{buys\_computer = yes}\\
\subsection*{Vấn đề thực tế khi áp dụng xác suất Bayes}
Kham khảo \cite{2WAYSOFNB}, khi khảo sát thực tế, biến hoặc giá trị của biến rất hiếm khi được phân loại trong khi giải thuật xác suất Bayes $\mathbb{P}(B|A) = \frac{\mathbb{P}(B) \cdot \mathbb{P}(A|B)}{\mathbb{P}(A)}$ làm việc với biến và giá trị biến được phân loại. Ví dụ như trong bài toán Luận Văn này, xem xét những bước di chuyển trong khoảng thời gian đều 20 giây trong 80\% tổng thời gian di chuyển để phán đoán xe có về đích đúng giờ hay không. Như vậy, biến trong bài toán này chính là những bước di chuyển chưa được phân loại, ví dụ phân loại thành những bước di chuyển rất nhỏ, nhỏ, trung bình, cao, rất cao. Đồng thời, sau khi phân loại thành các bước di chuyển thì số lần lặp lại những bước di chuyển đó đều là các con số liên tục, chưa được phân loại, ví dụ phân loại số lần lặp rất nhỏ, nhỏ, trung bình, lớn, rất lớn. Như vậy, trên thực tế, biến hoặc giá trị của biến rất hiếm khi được phân loại sẵn để thực hiện giải tương tự như ví dụ trên.\\
Bảng số liệu sau mô tả giá trị của biến trong bài toán Luận Văn được định lượng là các con số liên tục.
\begin{flushleft}
\begin{tabular}{|p{2.6cm}|p{2.6cm}|p{2.6cm}|p{2.6cm}|p{2.6cm}|p{1.5cm}|}
\hline
Số lần thực hiện bước đi từ 0-15 (km/h)& Số lần thực hiện bước đi từ 15-30 (km/h) & Số lần thực hiện bước đi từ 30-45 (km/h) & Số lần thực hiện bước đi từ 45-60 (km/h) & Số lần thực hiện bước đi trên 60 km/h & Trạng thái đến đích \\ 
\hline
19 & 23 & 19 & 29 & 3 & đúng giờ\\
\hline
16 & 14 & 40 & 21 & 0 & đúng giờ\\
\hline
19 & 23 & 48 & 13 & 0 & đúng giờ\\
\hline
14 & 27 & 43 & 12 & 0 & đúng giờ\\
\hline
13 & 33 & 44 & 9 & 0 & đúng giờ \\
\hline
45 & 26 & 25 & 25 & 0 & trễ giờ\\
\hline
56 & 26 & 27 & 16 & 0 & trễ giờ\\
\hline
29 & 22 & 41 & 22 & 1 & trễ giờ\\
\hline
83 & 11 & 15 & 12 & 5 & trễ giờ\\
\hline
28 & 32 & 36 & 21 & 0 & trễ giờ\\
\hline
\end{tabular}
\end{flushleft}
Có hai hướng giải quyết đối với thuộc tính ở dạng con số để cuối cùng vẫn áp dụng được giải thuật xác suất Bayes $\mathbb{P}(B|A) = \frac{\mathbb{P}(B) \cdot \mathbb{P}(A|B)}{\mathbb{P}(A)}$\\
\begin{enumerate}
\item Cách đơn giản nhất là rời rạc hóa thuộc tính dạng con số thành dạng phân loại. Tuy nhiên, sẽ có những tranh cãi khi phân loại, ví dụ với con số 29 lần xuất hiện nên phân loại là xuất hiện trung bình hay nhiều lần. Đồng thời, khi phân loại không tốt, có xảy ra hiện tượng hai dòng có cùng mẫu giống nhau nhưng kết quả cuối lại khác nhau. Ví dụ, cùng mẫu low,medium,medium,low,very low tượng trưng với dãy số 15,30,40,15,0 và 24,31,45,12,0 cho hai kết quả đúng giờ và trễ giờ\\ 
\begin{flushleft}
\begin{tabular}{|p{2.6cm}|p{2.6cm}|p{2.6cm}|p{2.6cm}|p{2.6cm}|p{1.5cm}|}
\hline
Số lần thực hiện bước đi từ 0-15 (km/h)& Số lần thực hiện bước đi từ 15-30 (km/h) & Số lần thực hiện bước đi từ 30-45 (km/h) & Số lần thực hiện bước đi từ 45-60 (km/h) & Số lần thực hiện bước đi trên 60 km/h & Trạng thái đến đích \\ 
\hline
15 (low) & 30 (medium) & 40 (medium) & 15 (low) & 0 (very low) & đúng giờ\\
\hline
24 (low) & 31 (medium) & 45 (medium) & 12 (low) & 0 (very low) & trễ giờ\\
\hline
\end{tabular}
\end{flushleft}
\item Sử dụng hàm mật độ xác suất (probability density function pdf) cho biến liên tục. Thông thường mọi người chọn giá trị liên tục của biến đầu vào có hàm mật độ xác suất tuân theo phân phối chuẩn hoặc phân phối Gaussian. Bạn vẫn có thể chọn phân phối khác, ví dụ như phân phối Poisson, phân phối Logarithmic,.., nếu như dữ liệu của bạn tuân theo phân phối đó.    
\end{enumerate}
\section{Phân phối chuẩn (Gauss)}
Kham khảo trang 87 đến trang 92 từ sách \cite{TKCNUDR}\\
Phân phối Gauss (hay phân phối chuẩn) có ký hiệu là $\textbf{N}(\mu,\sigma^2)$ chiếm một vai trò trung tâm trong lý thuyết thống kê. Hàm mật độ của $\textbf{N}(\mu,\sigma^2)$ được cho bởi công thức
\[
f(x)=n(x,\mu,\sigma)=\frac{1}{\sigma\sqrt{2\pi}} * e^{-\frac{(x-\mu)^2}{2\sigma^2}} \qquad \qquad \qquad -\infty < x < \infty, \mu \in \mathbb{R}, \sigma^2 > 0
\]
Biến ngẫu nhiên $\mathit{X}$ có hàm mật độ là hàm Gauss $f(x)$, xem hình dưới đây, thì ta nói X có phân phối chuẩn. Ký hiệu $\mathit{X} \thicksim \textbf{N}(\mu,\sigma)$\\
\FloatBarrier
\begin{figure}[h]
  \centering
    \includegraphics[width=0.5\textwidth]{Normal_Distribution_PDF.png}    
    \caption{Hàm mật độ của $\textbf{N}(\mu,\sigma)$ với $\mu=10, \sigma=1,2,3$}
\end{figure}
\FloatBarrier
\FloatBarrier
\begin{figure}[h]
  \centering
    \includegraphics[width=0.5\textwidth]{Empirical_Rule.png}    
    \caption{Hàm mật độ của $\textbf{N}(\mu,\sigma)$}
\end{figure}
\FloatBarrier
Đồ thị của $f(x)$ là một đường cong hình chuông đối xứng qua đường thẳng $\mathit{X} = \mu$. Sự phân phối của hàm mật độ được xác định bởi phương sai $\sigma^2$, theo nghĩa là hầu hết diện tích đường cong $f(x)$ (rõ hơn là 99.7\% diện tích) ở giữa hai đường thẳng $\mu-3\sigma$, $\mu+3\sigma$. Khi chúng ta di chuyển xa khỏi giá trị trung bình trong cả hai hướng, đường cong chuẩn tiếp cận trục hoành.\\
Biến ngẫu nhiên $\mathit{Z}$ tuân theo phân phối Gauss chuẩn tắc khi hàm mật độ của $\mathit{Z}$ là hàm số Gauss
\[
f(x) = n(x,0,1)=\frac{1}{\sqrt{2\pi}} * e^{-\frac{(x)^2}{2}}
\] 
Ký hiệu $\mathit{Z} \thicksim \textbf{N}(0,1)$. Giá trị kỳ vọng và phương sai của $\mathit{Z}$ tương ứng là $\textbf{E}(\mathit{Z})$ = 0 và $\textbf{V}(\mathit{Z})$ = 1. Hàm phân phối của biến $\mathit{Z} \thicksim \textbf{N}(0,1)$ là
\[
\Phi(x)= F(x) = \mathbb{P}[Z \leq x] = \int_{-\infty}^{x} f(t)d(t)
\] 
\FloatBarrier
\begin{figure}[h]
  \centering
    \includegraphics[width=0.5\textwidth]{standardNormalCDF.png}    
    \caption{Hàm tích lũy xác suất $\Phi(x)$ của phân phối Gauss chuẩn}
\end{figure}
\FloatBarrier
Hàm $\Phi(x)$ hình trên còn được gọi là hàm phân phối Gauss chuẩn tắc, cho ta diện tích khu vực trên trục $x$, bên dưới đường cong hàm mật độ của $\mathit(Z)$ và bên trái của giá trị $x$.\\
Việc tính diện tích (xác suất) phía dưới đường cong $f(x)$ của một biến $\mathit{X} \thicksim \textbf{N}(\mu,\sigma^2)$ cho mỗi cặp tham số $\mu$ và $\sigma$ là không thực tế. Ta có thể sử dụng biến chuẩn hóa - phép biến đổi $\mathit{Z}$ của X, cho bởi
\[
Z = \frac{X-\mu}{\sigma}
\]
Khi đó $\mathit{Z} \thicksim \textbf{N}(0,1)$.
\section{Sơ lược về phân loại Gaussian Bayes}
\underline{Định nghĩa không chính thống}:\\
Gọi X là biến đầu vào\\
Gọi Y là biến phân loại lớp 0 hoặc 1, có xác suất $\mathrm{P_Y}(0)$=$\mathrm{P_Y}(1)$=$\frac{1}{2}$\\
Công thức hàm phân phối Gaussian 
$
\mathrm{G}(x,\mu,\sigma)=\frac{1}{\sigma\sqrt{2\pi}} * e^{-\frac{(x-\mu)^2}{2\sigma^2}}  
$\\
với $\mu=\frac{\sum_1^n x_i}{n}$ và $\sigma^2=\frac{\sum_1^n (x_i-\mu)^2}{n}$\\
Biến X có hàm phân phối Gaussian khác nhau theo mỗi phân loại, nghĩa là \\
$\mathrm{P_{X|Y}}(x|0)=\mathrm{G}(x,\mu_0,\sigma_0)$\\
$\mathrm{P_{X|Y}}(x|1)=\mathrm{G}(x,\mu_1,\sigma_1)$\\
Hình vẽ trực quan\\  
\FloatBarrier
\begin{figure}[h]
  \centering
    \includegraphics[width=0.5\textwidth]{GaussBayesEx.png}    
    \caption{Biến X có hàm mật độ phân phối Gaussian khác nhau theo mỗi phân loại}
\end{figure}
\FloatBarrier 
Hình vẽ trên cho ta nhận xét, nếu x < $\frac{\mu_1+\mu_2}{2}$ thì phân loại x vào 0, nếu x > $\frac{\mu_1+\mu_2}{2}$ thì phân loại x vào 1.\\  
\FloatBarrier
\begin{figure}[h]
  \centering
    \includegraphics[width=0.5\textwidth]{GaussBayesEx2.png}
	\caption{Nếu x < $\frac{\mu_1+\mu_2}{2}$ thì phân loại x vào 0. Nếu x > $\frac{\mu_1+\mu_2}{2}$ thì phân loại x vào 1}
\end{figure}
\FloatBarrier 
\underline{Ví dụ về phân loại Gaussian Bayes}\\
Bài toán phân loại đứa trẻ (viết tắt c) hay người lớn (viết tắt a) dựa vào hai biến chiều cao[cm] (viết tắt h), cân nặng[kg] (viết tắt w). Có 4 dữ liệu adult và 12 dữ liệu child.\\
Ta có $\mathrm{P}(a)=\frac{4}{4+12}=0.25$, $\mathrm{P}(c)=\frac{12}{4+12}=0.75$\\
Công thức hàm phân phối Gaussian 
$
\mathrm{G}(x,\mu,\sigma)=\frac{1}{\sqrt{2\pi\sigma^2}} * e^{-\frac{(x-\mu)^2}{2\sigma^2}}  
$ \\
Với giả thiết, biến chiều cao và cân nặng độc lập nhau. Lập phân phối Gaussian cho từng biến thuộc từng phân loại\\
\begin{enumerate}
\item Hàm mật độ phân phối Gaussian cho phân loại người lớn
\begin{enumerate}
\item Biến chiều cao có phân phối Gaussian $\mathrm{G}(\mu_{h,a}, \sigma^2_{h,a})$ với $\mu_{h,a}=\frac{\sum_{i:y_i=a} h_i}{4}$, $\sigma^2_{h,a}=\frac{\sum_{i:y_i=a} (h_i-\mu_{h,a})^2}{4}$
\item Biến cân nặng có phân phối Gaussian $\mathrm{G}(\mu_{w,a}, \sigma^2_{w,a})$ với $\mu_{w,a}=\frac{\sum_{i:y_i=a} w_i}{4}$, $\sigma^2_{w,a}=\frac{\sum_{i:y_i=a} (w_i-\mu_{w,a})^2}{4}$
\end{enumerate}
\item Hàm mật độ phân phối Gaussian cho phân loại đứa trẻ
\begin{enumerate}
\item Biến chiều cao có phân phối Gaussian $\mathrm{G}(\mu_{h,c}, \sigma^2_{h,c})$ với $\mu_{h,c}=\frac{\sum_{i:y_i=c} h_i}{4}$, $\sigma^2_{h,c}=\frac{\sum_{i:y_i=c} (h_i-\mu_{h,c})^2}{4}$
\item Biến cân nặng có phân phối Gaussian $\mathrm{G}(\mu_{w,c}, \sigma^2_{w,c})$ với $\mu_{w,c}=\frac{\sum_{i:y_i=c} w_i}{4}$, $\sigma^2_{w,c}=\frac{\sum_{i:y_i=c} (w_i-\mu_{w,c})^2}{4}$
\end{enumerate}
\end{enumerate}
Cho dữ liệu kiểm tra x có giá trị chiều cao $x_h$, giá trị cân nặng $x_w$. Hỏi nên phân loại x vào đứa trẻ hay người lớn?\\
Để trả lời câu hỏi trên, ta cần so sánh hai xác suất $\mathrm{P}(a|x)$ và $\mathrm{P}(c|x)$, xác suất nào lớn hơn thì phân loại x vào lớp đó.\\
Ta có:
$
\mathrm{P}(a|x)=\frac{\mathrm{P}(x|a)\mathrm{P}(a)}{\mathrm{P}(x|a)\mathrm{P}(a)+\mathrm{P}(x|c)\mathrm{P}(c)}
$
và
$
\mathrm{P}(c|x)=\frac{\mathrm{P}(x|c)\mathrm{P}(c)}{\mathrm{P}(x|a)\mathrm{P}(a)+\mathrm{P}(x|c)\mathrm{P}(c)}
$\\
Vì mẫu số giống nhau, ta chỉ cần so sánh tử số $\mathrm{P}(x|a)\mathrm{P}(a)$ và $\mathrm{P}(x|c)\mathrm{P}(c)$\\
Ta đã có $\mathrm{P}(a)=0.25$, $\mathrm{P}(c)=0.75$.\\
Ta cần tính\\
$\mathrm{P}(x|a)$=$\mathrm{P}(h_x|a)\mathrm{P}(w_x|a)$ với $
\mathrm{P}(h_x|a)=\frac{1}{\sqrt{2\pi\sigma^2_{h,a}}} * e^{-\frac{(h_x-\mu_{h,a})^2}{2\sigma_{h,a}^2}}$, $
\mathrm{P}(w_x|a)=\frac{1}{\sqrt{2\pi\sigma^2_{w,a}}} * e^{-\frac{(w_x-\mu_{w,a})^2}{2\sigma_{w,a}^2}}$\\
và
$\mathrm{P}(x|c)$=$\mathrm{P}(h_x|c)\mathrm{P}(w_x|c)$ với $
\mathrm{P}(h_x|c)=\frac{1}{\sqrt{2\pi\sigma^2_{h,c}}} * e^{-\frac{(h_x-\mu_{h,c})^2}{2\sigma_{h,c}^2}}$, $
\mathrm{P}(w_x|c)=\frac{1}{\sqrt{2\pi\sigma^2_{w,c}}} * e^{-\frac{(w_x-\mu_{w,c})^2}{2\sigma_{w,c}^2}}$\\

\section{Thuật toán Kernel Density Estimation}
\underline{Định nghĩa hàm Kernel}:\\
Hàm kernel có các thuộc tính giống hàm mật độ xác suất (pdf) với một thuộc tính được cộng thêm là hàm kernel bắt buộc phải hàm chẵn (even function), nghĩa là với mọi giá trị $x$ và -$x$ trong miền trục hoành của f, ta có: $f(x) = f(-x)$. Nói cách khác, hàm chẵn đối xứng qua trục tung y, ví dụ như |$x$|, $x^2$, cos($x$). Với sự thuận tiện trong Toán học, phân phối chuẩn thường được dùng là hàm kernel.\\
\underline{Định nghĩa thuật toán Kernel Density Estimation (KDE)}:\\
Trong thống kê, thuật toán Kernel Density Estimation (KDE) là phương pháp không tham số (non-parametric method) dự đoán hàm mật độ xác suất pdf của biến ngẫu nhiên liên tục. KDE là phương pháp cơ bản để làm trơn dữ liệu trên mẫu dữ liệu giới hạn để dựa vào đó suy luận tổng thể.\\
\underline{Phương pháp không tham số}:\\
Không tham số không có nghĩa mô hình phân phối thiếu tham số mà tham số (ví dụ như trung bình, độ lệch chuẩn) và số lượng tham số không được xác định trước mà thay đổi và được xác định bởi dữ liệu. Ngoài ra, nó cũng là phương pháp không giả định dữ liệu thuộc về phân phối cụ thể được biết trước mà được quyết định được dữ liệu.\\
\underline{Từng bước thực hiện thuật toán Kernel Density Estimation (KDE)}:\\
Những bước tạo thành KDE:\\
\begin{enumerate}
\item Chọn hàm kernel, thường chọn phân phối chuẩn (Gaussian), phân phối đều (hình chữ nhật hay hình tam giác)
\item Tại mỗi điểm dữ liệu $x_i$, xây dựng hàm kernel thu nhỏ (scaled kernel) $K_h = \frac{1}{h}K[\frac{(x-x_i)}{h}]$, trong đó K là hàm kernel được chọn, tham số h được gọi là bandwidth, hoặc tham số làm trơn. 
\item Cộng tất cả các hàm kernel thu nhỏ và chia cho n, điều này thể hiện xác suất $\frac{1}{n}$ đối với mỗi $x_i$. Đồng thời, điều này đảm bảo rằng tích phân KDE trên miền giá trị x bằng 1.\\  
\[
\hat{f}(x_i) = \hat{p}_\textit{\tiny\tiny KDE}(x_i) = \frac{1}{n}\sum_{i=1}^{n} K_h = \frac{1}{n}\frac{1}{h}\sum_{i=1}^{n} K(\frac{x-x_i}{h})
\]
\end{enumerate}
Chú ý bước 2, tại mỗi điểm dữ liệu, hàm kernel thu nhỏ được tạo ra sao cho mỗi điểm dữ liệu là trung tâm của hàm, điều này bảo đảm hàm kernel đối xứng qua điểm dữ liệu. Hàm pdf dự đoán bằng cách cộng các hàm kernel thu nhỏ sao đó chia cho số lượng dữ liệu để bảo đảm hai tính chất của hàm mật độ xác suất pdf $\hat{p}_\textit{\tiny\tiny KDE}$(x), cụ thể: $\hat{p}_\textit{\tiny\tiny KDE}$(x) > 0 và $\int_{-\infty}^{\infty} \hat{p}_\textit{\tiny\tiny KDE}(x)dx = 1$.\\
Ví dụ để làm rõ điều trên, ta nhìn hình vẽ dưới đây:
\FloatBarrier
\begin{figure}[!htb]
\minipage{0.5\textwidth}
  \includegraphics[width=\linewidth]{500px-Comparison_of_1D_histogram_and_KDE}
\endminipage
\minipage{0.5\textwidth}
  \includegraphics[width=\linewidth]{KDE1}
\endminipage
\end{figure}
\FloatBarrier
Xét tại 6 điểm dữ liệu $x_1$ = -2.1, $x_2$ = -1.3, $x_3$ = -0.4, $x_4$ = 1.9, $x_5$ = 5.1, $x_6$ = 6. Đối với histogram (hình vẽ bên trái), sẽ có 6 khoảng nhỏ, mỗi khoảng có kích thước chiều rộng 2, chiều cao tùy thuộc vào số lượng điểm dữ liệu rơi vào khoảng đó, cứ mỗi điểm rơi vào thì khoảng có có chiều cao cộng thêm 1/12. Đối với KDE (hình vẽ bên phải), ta sử dụng phân phối chuẩn với độ sai lệch (variance 2.25) trên mỗi điểm dữ liệu $x_i$ (được thể hiện đường màu đỏ). Cộng lại các kernel này tạo ra KDE (được thể hiện đường màu xanh). KDE khá giống như histogram, ngoại trừ có thêm tính liên tục hay tính trơn nhờ sử dụng kernel thích hợp.\\  
\underline{Chọn băng thông}:\\
Việc chọn bandwidth là một việc khó để giải thuật KDE thể hiện tốt phân phối của biến, và sẽ không viết chi tiết ở đây. Chọn bandwidth h lớn thì làm trơn quá và che phủ cấu trúc dữ liệu, trong khi bandwidth h nhỏ thì hình dạng phân phối có nhiều gai nhọn và rất khó để thông dịch. Như hình dưới đây:
\FloatBarrier
\begin{figure}[!h]
  \begin{subfigure}[b]{0.7\textwidth}
	   \label{tab:example}
        \includegraphics[width=\linewidth]{KDE2}
  \end{subfigure}%
\end{figure}
\FloatBarrier
Bandwidth của kernel là tham số tự do, nhưng có ảnh hưởng lớn đến hiệu quả dự đoán. Ví dụ, lấy mẫu ngẫu nhiên (những đường vạch màu xanh da trời trên trục ngang) biết trước có phân phối chuẩn thể hiện màu xám có trung bình mẫu bằng 0 và độ sai lệch bằng 1 (biểu diễn bởi đường xám). Với bandwidth h = 0.05 có kernel thể hiện đường cong màu đỏ, hoặc với bandwidth h = 2 có kernel thể hiện màu xanh lá đều dự đoán lệch quá xa so với kết quả biết trước, gọi đường kernel màu đỏ làm trơn quá mức (over smooth) và gọi đường kernel màu xanh lá làm trơn dưới mức (under smooth), trong khi với bandwidth h = 0.337 có kernel thể hiện đường màu đen có dự đoán gần giống với kết quả, được gọi là làm trơn tối ưu (optimal smooth).\\
\FloatBarrier
\begin{figure}[!h]
  \begin{subfigure}[b]{0.5\textwidth}
	   \label{tab:example}
        \includegraphics[width=\linewidth]{220px-Comparison_of_1D_bandwidth_selectors}
  \end{subfigure}%
\end{figure}
\FloatBarrier
MISE (mean integrated squared error) là tiêu chuẩn thông thường sử dụng để lựa chọn tham số bandwidth:
\[
MSIE = E[\int(\hat{f}(x)-f(x))^2dx] = E[\int(\hat{p}_\textit{\tiny\tiny KDE}(x) - p(x))^2dx] = 
\underbrace{E[\int(\hat{p}_\textit{\tiny\tiny KDE}(x) - p(x))dx]^2}_{\text{bias}} + 
\underbrace{var(\hat{p}_\textit{\tiny\tiny KDE}(x))}_{\text{variance}}
\]
Bias còn được gọi là lỗi hệ thống, lỗi được gây khi tiến hành nghiên cứu ví dụ như chọn hay loại bỏ đối tượng được nghiên cứu, do lỗi thiết bị hay đo đạc, phân loại sai, hoặc thậm chí cách thức được chọn nghiên cứu không đúng hướng ngay từ đâu.\\ Variance được gọi là lỗi mang tính chất ngẫu nhiên, không dự đoán và không biết trước để tránh và có thể không lặp lại nếu tiếp tục thí nghiệm một lần nữa, khác với bias, lỗi sẽ lặp lại như vậy nếu vẫn tiến hành theo cách cũ. Chọn bandwidth h tốt là làm nhỏ nhất giá trị MISE.\\
Khi chọn bandwidth h lớn làm giảm lỗi ngẫu nhiên, nhưng làm tăng lỗi hệ thống. Trong khi bandwidth h nhỏ làm giảm lỗi hệ thống, nhưng làm tăng lỗi ngẫu nhiên. Hai hình vẽ sau thể hiện điều này 
\FloatBarrier
\begin{figure}[!h]
  \begin{subfigure}[b]{0.5\textwidth}
	   \label{tab:example}
        \includegraphics[width=\linewidth]{KDE3}
  \end{subfigure}%
\end{figure}
\FloatBarrier
Nếu ta giả sử dữ liệu nghiên cứu tuân theo phân phối chuẩn và ta chọn Gaussian làm kernel cho giải thuật KDE, thì ta có thể biết giá trị tối ưu của h là $h^1$ = 1.06$\sigma N^{-\frac{1}{5}}$.
\chapter{HIỆN THỰC VÀ THỬ NGHIỆM}
\section{Dữ liệu nghiên cứu}
\subsection{Dữ liệu thô}
Dữ liệu thô là tọa độ vĩ độ, kinh độ và thời điểm xuất hiện tại vị trí đó của xe buýt\\
Ví dụ 10 dữ liệu thô:\\
\begin{flushleft}
\begin{tabular}{ |l|l|l|l| }
\hline
&Vĩ độ & Kinh độ & Thời điểm xuất hiện \\ 
\hline
1&10.844095 & 106.613688333333 & 2016-09-02 07:25:43 \\ 
\hline
2&10.84298 & 106.614991666667 & 2016-09-02 07:26:02 \\
\hline
3&10.8424316666667 & 106.615195 & 2016-09-02 07:26:22 \\
\hline
4&10.8426816666667 & 106.615596666667 & 2016-09-02 07:26:42 \\
\hline
5&10.84309 & 106.615203333333 & 2016-09-02 07:27:02 \\
\hline
6&10.846395 & 106.61304 & 2016-09-02 07:30:48 \\
\hline
7&10.84664 & 106.612861666667 & 2016-09-02 07:31:01 \\
\hline
8&10.8475833333333 & 106.612253333333 & 2016-09-02 07:31:21 \\
\hline
9&10.8488916666667 & 106.611426666667 & 2016-09-02 07:31:41 \\
\hline
10&10.84932 & 106.61117 & 2016-09-02 07:31:53 \\
\hline 
\end{tabular}
\end{flushleft}
\subsection{Tiền xử lý dữ liệu} 
\subsubsection{Kiểm tra lộ trình xe buýt}
Kham khảo \ref{DragAndDrop} tại mục Phụ Lục để lọc ra và kiểm tra dữ liệu đang khảo sát có phải là tuyến đi BX Củ Chi - BX An Sương không.\\
\subsubsection{Chọn thời gian hoàn thành một chuyến đi}
Theo quy định của quản lý xe, thời gian hoàn thành BX Củ Chi - BX An Sương là 45 phút, cho nên ta chỉ chọn những chuyến xe hoàn thành từ khoảng 30 phút đến 60 phút, những xe vượt quá thời gian hoàn thành trên 1 tiếng bị loại bỏ.\\ 
Kham khảo \ref{histogramWithNormalCurve} tại mục Phụ Lục xuất ra hình vẽ phân phối tần suất thời gian hoàn thành lộ trình BX Củ Chi - BX An Sương\\
\FloatBarrier
\begin{figure}[h!]
        \begin{subfigure}[b]{0.5\textwidth}
                \label{tab:example}
                \includegraphics[width=\linewidth]{histogramCCAS}
        \end{subfigure}%
\end{figure}
\FloatBarrier
Kham khảo \ref{sapxepTGHT} tại mục Phụ Lục xuất ra hình vẽ sắp xếp thời gian hoàn thành lộ trình BX Củ Chi - BX An Sương xác suất xuất hiện tăng dần.\\
\FloatBarrier
\begin{figure}[h!]
        \begin{subfigure}[b]{0.7\textwidth}
        		\label{tab:example2}
                \includegraphics[width=\linewidth]{finishTime_CC_AS_Order}
        \end{subfigure}%
\end{figure}
\FloatBarrier
Kham khảo \ref{xacsuatTLTGHT} tại mục Phụ Lục xuất ra hình vẽ cho thấy xác suất tích lũy thời gian hoàn thành lộ trình BX Củ Chi - BX An Sương \\
\FloatBarrier
\begin{figure}[h!]
        \begin{subfigure}[b]{0.7\textwidth}
        		\label{tab:example2}
                \includegraphics[width=\linewidth]{finishTime_CC_AS_ACC_Line}
        \end{subfigure}%
\end{figure}
\FloatBarrier
Ta chọn thời gian hoàn thành một chuyến đi từ BX Củ Chi đến BX An Sương trước đúng 45 phút là về đích đúng giờ, và từ hình vẽ trên ta thấy có 84\% trong 1289 chuyến xe được khảo sát đạt được thời gian về đích là 45 phút.\\
Như vậy ta có $\mathrm{P}$(về đích đúng giờ) = 0.84 và $\mathrm{P}$(về đích trễ giờ) = 0.16
\subsubsection{Đồng bộ hóa ghi nhận mỗi bước di chuyển}
Sau đó kham khảo \ref{tinhkhoangcach} tại mục Phụ Lục để tính mỗi bước di chuyển. Sau đây là ví dụ 10 bước di chuyển đầu tiên trong một chuyến đi ngẫu nhiên được chọn\\
\begin{flushleft}
\begin{tabular}{|p{1cm}|p{1cm}|p{2.6cm}|p{2.6cm}|p{2.6cm}|p{2.6cm}|p{1cm}|p{1cm}| }
\hline
Bước đầu & Bước kế tiếp & Vĩ độ đầu & Kinh độ đầu & Vĩ độ kế tiếp & Kinh độ kế tiếp & Khoảng cách (mét) & Thời gian di chuyển (giây) \\ 
\hline
0 & 1 & 10.971375 & 106.481906666667 & 10.9710116666667 & 106.481858333333 & 41 & 14 \\
\hline
1 & 2 & 10.9710116666667 & 106.481858333333 & 10.9708933333333 & 106.482023333333 & 22 & 13 \\
\hline
2 & 3 & 10.9708933333333 & 106.482023333333 & 10.9708166666667 & 106.482206666667 & 22 & 06 \\
\hline
3 & 4 & 10.9708166666667 & 106.482206666667 & 10.9706533333333 & 106.482398333333 & 28 & 13 \\
\hline
4 & 5 & 10.9706533333333 & 106.482398333333 & 10.970555 & 106.48252 & 17 & 11 \\
\hline
5 & 6 & 10.970555 & 106.48252 & 10.9701433333333 & 106.483123333333 & 80 & 20 \\
\hline
6 & 7 & 10.9701433333333 & 106.483123333333 & 10.9695933333333 & 106.48386 & 101 & 20 \\
\hline
7 & 8 & 10.9695933333333 & 106.48386 & 10.968885 & 106.485015 & 149 & 20 \\
\hline
8 & 9 & 10.968885 & 106.485015 & 10.9682416666667 & 106.48598 & 127 & 20 \\
\hline
9 & 10 & 10.9682416666667 & 106.48598 & 10.9677616666667 & 106.487061666667 & 130 & 20 \\
\hline
10 & 11 & 10.9677616666667 & 106.487061666667 & 10.9674233333333 & 106.487998333333 & 109 & 20 \\
\hline 
\end{tabular}
\end{flushleft}
Khi quan sát, ta nhận thấy thông thường sau 20 giây hệ thống sẽ ghi nhận một bước chuyển mới, nhưng do trục trặc về kỹ thuật ghi nhận, vẫn thường xuyên xảy ra hiện tượng hồi đáp sớm hơn, trễ hơn so với 20 giây, cho nên ta phải làm một thao tác trung gian đồng bộ hóa 20 giây nhiều nhất có thể. Vì ta giả định những bước di chuyển hoàn toàn độc lập nhau nên ta có thể gom nhóm những bước hồi đáp không chuẩn lại thành bội số của 20 và sau đó chia đều cho 20, nếu số dư còn lại lớn hơn 15 giây thì chấp nhận hồi đáp dư đó, nếu không thì loại bỏ.\\      
Kham khảo \ref{dongbohoa20} tại mục Phụ lục cho thao tác đồng bộ hóa 20 giây và ví dụ sau xem kết quả đồng bộ cho một chuyến được chọn ngẫu nhiên\\
Dữ liệu chưa được đồng bộ hóa
\FloatBarrier
\begin{table}[!htb]
\begin{minipage}{.3\linewidth}
\begin{tabular}{ |l|p{1cm}|p{1cm}| }
\hline
STT&Khoảng thời gian hồi đáp (giây) & Khoảng cách bước đi (mét)\\
\hline
\hline
1&17&236\\
\hline
2&20&71\\
\hline
3&20&86\\
\hline
4&20&145\\
\hline
5&20&136\\
\hline
6&9&93\\
\hline
7&1&9\\
\hline
8&20&104\\
\hline
9&20&277\\
\hline
10&20&240\\
\hline
11&5&6\\
\hline
12&12&11\\
\hline
13&20&145\\
\hline
14&20&97\\
\hline
15&14&17\\
\hline
16&20&0\\
\hline
17&2&14\\
\hline
18&20&171\\
\hline
19&20&283\\
\hline
20&20&283\\
\hline
\end{tabular}
\end{minipage}%
\begin{minipage}{.3\linewidth}
\begin{tabular}{ |l|p{1cm}|p{1cm}| }
\hline
STT&Khoảng thời gian hồi đáp (giây) & Khoảng cách bước đi (mét)\\
\hline
\hline
21&21&106\\
\hline
22&5&3\\
\hline
23&20&0\\
\hline
24&7&17\\
\hline
25&20&214\\
\hline
26&15&115\\
\hline
27&10&13\\
\hline
28&20&166\\
\hline
29&20&307\\
\hline
30&20&259\\
\hline
31&20&208\\
\hline
32&20&43\\
\hline
33&20&144\\
\hline
34&20&286\\
\hline
35&20&287\\
\hline
36&20&126\\
\hline
37&20&152\\
\hline
38&20&254\\
\hline
39&18&111\\
\hline
40&7&9\\
\hline
\end{tabular}
\end{minipage} 
\begin{minipage}{.3\linewidth}
\begin{tabular}{ |l|p{1cm}|p{1cm}| }
\hline
STT&Khoảng thời gian hồi đáp (giây) & Khoảng cách bước đi (mét)\\
\hline
\hline
41&20&139\\
\hline
42&20&255\\
\hline
43&20&294\\
\hline
44&20&293\\
\hline
45&20&269\\
\hline
46&20&283\\
\hline
47&20&267\\
\hline
48&20&221\\
\hline
49&14&68\\
\hline
50&6&12\\
\hline
51&20&165\\
\hline
52&20&266\\
\hline
53&20&259\\
\hline
54&20&220\\
\hline
55&20&168\\
\hline
56&13&19\\
\hline
57&20&0\\
\hline
58&15&32\\
\hline
59&12&32\\
\hline
60&4&17\\
\hline
\end{tabular}
\end{minipage}
\end{table}

\begin{table}[!htb]
\begin{minipage}{.3\linewidth}
\begin{tabular}{ |l|p{1cm}|p{1cm}| }
\hline
STT&Khoảng thời gian hồi đáp (giây) & Khoảng cách bước đi (mét)\\
\hline
\hline
61&20&226\\
\hline
62&20&337\\
\hline
63&20&394\\
\hline
64&20&224\\
\hline
65&11&39\\
\hline
66&5&10\\
\hline
67&20&208\\
\hline
68&20&340\\
\hline
69&20&359\\
\hline
70&20&259\\
\hline
71&20&297\\
\hline
72&20&325\\
\hline
73&20&303\\
\hline
74&20&302\\
\hline
75&20&284\\
\hline
76&20&287\\
\hline
77&20&268\\
\hline
78&20&199\\
\hline
79&11&17\\
\hline
80&6&12\\
\hline
\end{tabular}
\end{minipage}%
\begin{minipage}{.3\linewidth}
\begin{tabular}{ |l|p{1cm}|p{1cm}| }
\hline
STT&Khoảng thời gian hồi đáp (giây) & Khoảng cách bước đi (mét)\\
\hline
\hline
81&20&112\\
\hline
82&6&9\\
\hline
83&7&17\\
\hline
84&20&179\\
\hline
85&20&279\\
\hline
86&20&270\\
\hline
87&20&139\\
\hline
88&20&72\\
\hline
89&10&39\\
\hline
90&7&12\\
\hline
91&20&195\\
\hline
92&20&345\\
\hline
93&20&332\\
\hline
94&20&151\\
\hline
95&10&9\\
\hline
96&20&149\\
\hline
97&20&230\\
\hline
98&20&261\\
\hline
99&20&186\\
\hline
100&14&37\\
\hline
\end{tabular}
\end{minipage} 
\begin{minipage}{.3\linewidth}
\begin{tabular}{ |l|p{1cm}|p{1cm}| }
\hline
STT&Khoảng thời gian hồi đáp (giây) & Khoảng cách bước đi (mét)\\
\hline
\hline
101&19&20\\
\hline
102&20&170\\
\hline
103&20&236\\
\hline
104&20&243\\
\hline
105&20&265\\
\hline
106&18&51\\
\hline
107&7&14\\
\hline
108&20&151\\
\hline
109&20&201\\
\hline 
&&\\
\hline
&&\\
\hline
&&\\
\hline
&&\\
\hline
&&\\
\hline
&&\\
\hline
&&\\
\hline
&&\\
\hline
&&\\
\hline
&&\\
\hline
&&\\
\hline
\end{tabular}
\end{minipage}
\end{table}
\FloatBarrier
Dữ liệu sau khi được đồng bộ hóa\\
\FloatBarrier
\begin{table}[!htb]
\begin{minipage}{.3\linewidth}
\begin{tabular}{ |l|p{1cm}|p{1cm}| }
\hline
STT&Khoảng thời gian hồi đáp (giây) & Khoảng cách bước đi (mét)\\
\hline
\hline
1&20&71\\
\hline
2&20& 86\\
\hline
3&20& 145\\
\hline
4&20& 136\\
\hline
5&20& 104\\
\hline
6&20& 277\\
\hline
7&20& 240\\
\hline
8&20& 145\\
\hline
9&20& 97\\
\hline
10&20& 0\\
\hline
11&20& 171\\
\hline
12&20& 283\\
\hline
13&20& 283\\
\hline
14&20& 0\\
\hline
15&20& 214\\
\hline
16&20& 166\\
\hline
17&20& 307\\
\hline
18&20& 259\\
\hline
19&20& 208\\
\hline
20&20& 43\\
\hline
\end{tabular}
\end{minipage}%
\begin{minipage}{.3\linewidth}
\begin{tabular}{ |l|p{1cm}|p{1cm}| }
\hline
STT&Khoảng thời gian hồi đáp (giây) & Khoảng cách bước đi (mét)\\
\hline
\hline
21&20& 144\\
\hline
22&20& 286\\
\hline
23&20& 287\\
\hline
24&20& 126\\
\hline
25&20& 152\\
\hline
26&20& 254\\
\hline
27&20& 139\\
\hline
28&20& 255\\
\hline
29&20& 294\\
\hline
30&20& 293\\
\hline
31&20& 269\\
\hline
32&20& 283\\
\hline
33&20& 267\\
\hline
34&20& 221\\
\hline
35&20& 165\\
\hline
36&20& 266\\
\hline
37&20& 259\\
\hline
38&20& 220\\
\hline
39&20& 168\\
\hline
40&20& 0\\
\hline
\end{tabular}
\end{minipage} 
\begin{minipage}{.3\linewidth}
\begin{tabular}{ |l|p{1cm}|p{1cm}| }
\hline
STT&Khoảng thời gian hồi đáp (giây) & Khoảng cách bước đi (mét)\\
\hline
\hline
41&20& 226\\
\hline
42&20& 337\\
\hline
43&20& 394\\
\hline
44&20& 224\\
\hline
45&20& 208\\
\hline
46&20& 340\\
\hline
47&20& 359\\
\hline
48&20& 259\\
\hline
49&20& 297\\
\hline
50&20& 325\\
\hline
51&20& 303\\
\hline
52&20& 302\\
\hline
53&20& 284\\
\hline
54&20& 287\\
\hline
55&20& 268\\
\hline
56&20& 199\\
\hline
57&20& 112\\
\hline
58&20& 179\\
\hline
59&20& 279\\
\hline
60&20& 270\\
\hline
\end{tabular}
\end{minipage}
\end{table}

\begin{table}[!htb]
\begin{minipage}{.5\linewidth}
\begin{tabular}{ |l|p{1cm}|p{1cm}| }
\hline
STT&Khoảng thời gian hồi đáp (giây) & Khoảng cách bước đi (mét)\\
\hline
\hline
61&20& 139\\
\hline
62&20& 72\\
\hline
63&20& 195\\
\hline
64&20& 345\\
\hline
65&20& 332\\
\hline
66&20& 151\\
\hline
67&20& 149\\
\hline
68&20& 230\\
\hline
69&20& 261\\
\hline
70&20& 186\\
\hline
71&20& 170\\
\hline
72&20& 236\\
\hline
73&20& 243\\
\hline
74&20& 265\\
\hline
75&20& 151\\
\hline
76&20& 201\\
\hline
77&20& 29\\
\hline
78&20& 125\\
\hline
79&20& 121\\
\hline
80&20& 35\\
\hline
\end{tabular}
\end{minipage}%
\begin{minipage}{.5\linewidth}
\begin{tabular}{ |l|p{1cm}|p{1cm}| }
\hline
STT&Khoảng thời gian hồi đáp (giây) & Khoảng cách bước đi (mét)\\
\hline
\hline
81&20& 29\\
\hline
82&20& 80\\
\hline
83&20& 46\\
\hline
84&20& 36\\
\hline
85&20& 132\\
\hline
86&20& 52\\
\hline
87&20& 44\\
\hline
88&20& 63\\
\hline
89&20& 63\\
\hline
90&20& 46\\
\hline
91&20& 46\\
\hline
92&21& 40\\
\hline
93&17& 236\\
\hline
&&\\
\hline
&&\\
\hline
&&\\
\hline
&&\\
\hline
&&\\
\hline
&&\\
\hline
&&\\
\hline
\end{tabular}
\end{minipage} 
\end{table}
\FloatBarrier
\subsubsection{Trực quan những bước di chuyển}
Trích ngẫu nhiên 80\% thời gian di chuyển của 24 chuyến xe BX Củ Chi - BX An Sương. Trong một chuyến xe, mỗi giá trị tính bằng mét, mỗi bước di chuyển giữa các giá trị được ghi nhận sau 20 giây.\\
\textbf{\underline{Chuyến 1}}: 
71, 86, 145, 136, 104, 277, 240, 145, 97, 0, 171, 283, 283, 0, 214, 166, 307, 259, 208, 43, 144, 286, 287, 126, 152, 254, 139, 255, 294, 293, 269, 283, 267, 221, 165, 266, 259, 220, 168, 0, 226, 337, 394, 224, 208, 340, 359, 259, 297, 325, 303, 302, 284, 287, 268, 199, 112, 179, 279, 270, 139, 72, 195, 345, 332, 151, 149, 230, 261, 186, 170, 236, 243, 265, 151, 201, 29, 125, 121, 35, 29, 80, 46, 36, 132, 52, 44, 63, 63, 46, 46, 40, 236\\
\textbf{\underline{Chuyến 2}}:...\\
...\\
\textbf{\underline{Chuyến 12}}: 104, 145, 124, 166, 194, 205, 188, 158, 107, 121, 91, 152, 218, 93, 100, 0, 167, 167, 250, 160, 144, 223, 215, 131, 114, 241, 206, 193, 0, 0, 169, 212, 192, 200, 257, 223, 198, 166, 135, 184, 105, 90, 151, 202, 192, 168, 177, 217, 168, 243, 211, 109, 161, 272, 287, 281, 248, 257, 166, 261, 250, 152, 238, 267, 190, 111, 0, 117, 138, 209, 182, 123, 58, 151, 222, 248, 218, 110, 0, 0, 111, 211, 61, 114, 83, 109, 26, 232, 232, 232, 232, 232, 232, 232, 232, 232, 96, 37, 110, 48, 34, 121, 121, 46, 33, 88, 64, 235, 183, 125, 220\\
\textbf{\underline{Chuyến 13}}:...\\
...\\
\textbf{\underline{Chuyến 24}}: 279, 17, 52, 48, 67, 141, 131, 131, 95, 157, 155, 196, 242, 219, 50, 172, 219, 243, 148, 74, 5, 105, 194, 68, 216, 224, 231, 61, 76, 228, 295, 306, 293, 260, 236, 279, 284, 232, 174, 75, 205, 103, 98, 232, 266, 257, 188, 60, 192, 228, 256, 236, 215, 23, 10, 100, 44, 132, 255, 290, 272, 216, 237, 281, 274, 272, 294, 189, 0, 41, 84, 65, 197, 247, 221, 121, 83, 138, 252, 223, 195, 135, 17, 0, 6, 15, 44, 121, 61, 206, 220, 148, 52, 164, 208, 248, 203, 179, 11, 0, 62, 112, 252, 307, 316, 168, 104, 87, 192, 148, 131, 91, 215, 215, 143, 126, 279, 279\\
Kham khảo \ref{trucquandichuyen} tại mục Phụ Lục xuất ra trực quan hóa giá trị các bước di chuyển của 24 chuyến dữ liệu ngẫu nhiên, ta có\\

\FloatBarrier
\begin{figure}[!htb]
\minipage{0.25\textwidth}
  \includegraphics[width=\linewidth]{test_100_1}
  \caption*{Toàn bộ di chuyển chuyến xe 1 - đúng giờ}
\endminipage
\minipage{0.25\textwidth}
  \includegraphics[width=\linewidth]{test_80_1}
  \caption*{80\% di chuyển chuyến xe 1 - đúng giờ}
\endminipage
\minipage{0.25\textwidth}%
  \includegraphics[width=\linewidth]{test_100_2}
  \caption*{Toàn bộ di chuyển chuyến xe 2 - trễ giờ}
\endminipage
\minipage{0.25\textwidth}
  \includegraphics[width=\linewidth]{test_80_2}
  \caption*{80\% di chuyển chuyến xe 2 - trễ giờ}
\endminipage\\
\minipage{0.25\textwidth}
  \includegraphics[width=\linewidth]{test_100_3}
  \caption*{Toàn bộ di chuyển chuyến xe 3 - đúng giờ}
\endminipage
\minipage{0.25\textwidth}
  \includegraphics[width=\linewidth]{test_80_3}
  \caption*{80\% di chuyển chuyến xe 3 - đúng giờ}
\endminipage
\minipage{0.25\textwidth}%
  \includegraphics[width=\linewidth]{test_100_4}
  \caption*{Toàn bộ di chuyển chuyến xe 4 - trễ giờ}
\endminipage
\minipage{0.25\textwidth}
  \includegraphics[width=\linewidth]{test_80_4}
  \caption*{80\% di chuyển chuyến xe 4 - trễ giờ}
\endminipage\\
\minipage{0.25\textwidth}
  \includegraphics[width=\linewidth]{test_100_5}
  \caption*{Toàn bộ di chuyển chuyến xe 5 - đúng giờ}
\endminipage
\minipage{0.25\textwidth}
  \includegraphics[width=\linewidth]{test_80_5}
  \caption*{80\% di chuyển chuyến xe 5 - đúng giờ}
\endminipage
\minipage{0.25\textwidth}%
  \includegraphics[width=\linewidth]{test_100_6}
  \caption*{Toàn bộ di chuyển chuyến xe 6 - trễ giờ}
\endminipage
\minipage{0.25\textwidth}
  \includegraphics[width=\linewidth]{test_80_6}
  \caption*{80\% di chuyển chuyến xe 6 - trễ giờ}
\endminipage\\
\minipage{0.25\textwidth}
  \includegraphics[width=\linewidth]{test_100_7}
  \caption*{Toàn bộ di chuyển chuyến xe 7 - đúng giờ}
\endminipage
\minipage{0.25\textwidth}
  \includegraphics[width=\linewidth]{test_80_7}
  \caption*{80\% di chuyển chuyến xe 7 - đúng giờ}
\endminipage
\minipage{0.25\textwidth}%
  \includegraphics[width=\linewidth]{test_100_8}
  \caption*{Toàn bộ di chuyển chuyến xe 8 - trễ giờ}
\endminipage
\minipage{0.25\textwidth}
  \includegraphics[width=\linewidth]{test_80_8}
  \caption*{80\% di chuyển chuyến xe 8 - trễ giờ}
\endminipage\\
\minipage{0.25\textwidth}
  \includegraphics[width=\linewidth]{test_100_9}
  \caption*{Toàn bộ di chuyển chuyến xe 9 - đúng giờ}
\endminipage
\minipage{0.25\textwidth}
  \includegraphics[width=\linewidth]{test_80_9}
  \caption*{80\% di chuyển chuyến xe 9 - đúng giờ}
\endminipage
\minipage{0.25\textwidth}%
  \includegraphics[width=\linewidth]{test_100_10}
  \caption*{Toàn bộ di chuyển chuyến xe 10 - trễ giờ}
\endminipage
\minipage{0.25\textwidth}
  \includegraphics[width=\linewidth]{test_80_10}
  \caption*{80\% di chuyển chuyến xe 10 - trễ giờ}
\endminipage\\
\end{figure}
\FloatBarrier
\FloatBarrier
\begin{figure}[!htb]
\minipage{0.25\textwidth}
  \includegraphics[width=\linewidth]{test_100_11}
  \caption*{Toàn bộ di chuyển chuyến xe 11 - đúng giờ}
\endminipage
\minipage{0.25\textwidth}
  \includegraphics[width=\linewidth]{test_80_11}
  \caption*{80\% di chuyển chuyến xe 11 - đúng giờ}
\endminipage
\minipage{0.25\textwidth}%
  \includegraphics[width=\linewidth]{test_100_12}
  \caption*{Toàn bộ di chuyển chuyến xe 12 - trễ giờ}
\endminipage
\minipage{0.25\textwidth}
  \includegraphics[width=\linewidth]{test_80_12}
  \caption*{80\% di chuyển chuyến xe 12 - trễ giờ}
\endminipage\\
\minipage{0.25\textwidth}
  \includegraphics[width=\linewidth]{test_100_13}
  \caption*{Toàn bộ di chuyển chuyến xe 13 - đúng giờ}
\endminipage
\minipage{0.25\textwidth}
  \includegraphics[width=\linewidth]{test_80_13}
  \caption*{80\% di chuyển chuyến xe 13 - đúng giờ}
\endminipage
\minipage{0.25\textwidth}%
  \includegraphics[width=\linewidth]{test_100_14}
  \caption*{Toàn bộ di chuyển chuyến xe 14 - trễ giờ}
\endminipage
\minipage{0.25\textwidth}
  \includegraphics[width=\linewidth]{test_80_14}
  \caption*{80\% di chuyển chuyến xe 14 - trễ giờ}
\endminipage\\
\minipage{0.25\textwidth}
  \includegraphics[width=\linewidth]{test_100_15}
  \caption*{Toàn bộ di chuyển chuyến xe 15 - đúng giờ}
\endminipage
\minipage{0.25\textwidth}
  \includegraphics[width=\linewidth]{test_80_15}
  \caption*{80\% di chuyển chuyến xe 15 - đúng giờ}
\endminipage
\minipage{0.25\textwidth}%
  \includegraphics[width=\linewidth]{test_100_16}
  \caption*{Toàn bộ di chuyển chuyến xe 16 - trễ giờ}
\endminipage
\minipage{0.25\textwidth}
  \includegraphics[width=\linewidth]{test_80_16}
  \caption*{80\% di chuyển chuyến xe 16 - trễ giờ}
\endminipage\\
\minipage{0.25\textwidth}
  \includegraphics[width=\linewidth]{test_100_17}
  \caption*{Toàn bộ di chuyển chuyến xe 17 - đúng giờ}
\endminipage
\minipage{0.25\textwidth}
  \includegraphics[width=\linewidth]{test_80_17}
  \caption*{80\% di chuyển chuyến xe 17 - đúng giờ}
\endminipage
\minipage{0.25\textwidth}%
  \includegraphics[width=\linewidth]{test_100_18}
  \caption*{Toàn bộ di chuyển chuyến xe 18 - trễ giờ}
\endminipage
\minipage{0.25\textwidth}
  \includegraphics[width=\linewidth]{test_80_18}
  \caption*{80\% di chuyển chuyến xe 18 - trễ giờ}
\endminipage\\
\minipage{0.25\textwidth}
  \includegraphics[width=\linewidth]{test_100_19}
  \caption*{Toàn bộ di chuyển chuyến xe 19 - đúng giờ}
\endminipage
\minipage{0.25\textwidth}
  \includegraphics[width=\linewidth]{test_80_19}
  \caption*{80\% di chuyển chuyến xe 19 - đúng giờ}
\endminipage
\minipage{0.25\textwidth}%
  \includegraphics[width=\linewidth]{test_100_20}
  \caption*{Toàn bộ di chuyển chuyến xe 20 - trễ giờ}
\endminipage
\minipage{0.25\textwidth}
  \includegraphics[width=\linewidth]{test_80_20}
  \caption*{80\% di chuyển chuyến xe 20 - trễ giờ}
\endminipage\\
\caption*{Trực quan hóa 20 dòng dữ liệu ngẫu nhiên}
\end{figure}
\FloatBarrier
\section{Rời rạc hóa những bước di chuyển}
Thực hiện phân loại những bước di chuyển: biến $X_1$ là bước đi rất nhỏ 0-15 (km/h), biến $X_2$ là bước đi nhỏ 15-30 (km/h), biến $X_3$ là bước đi trung bình 30-45 (km/h), biến $X_4$ là bước đi xa 45-60 (km/h), biến $X_5$ là bước đi rất xa trên 60 km/h.\\
Bảng số liệu sau mô tả 10 dữ liệu mà ta sẽ làm việc.
\begin{flushleft}
\begin{tabular}{|p{2.6cm}|p{2.6cm}|p{2.6cm}|p{2.6cm}|p{2.6cm}|p{1.5cm}|}
\hline
Số lần thực hiện bước đi từ 0-15 (km/h)& Số lần thực hiện bước đi từ 15-30 (km/h) & Số lần thực hiện bước đi từ 30-45 (km/h) & Số lần thực hiện bước đi từ 45-60 (km/h) & Số lần thực hiện bước đi trên 60 km/h & Trạng thái đến đích \\ 
\hline
19 & 23 & 19 & 29 & 3 & đúng giờ\\
\hline
16 & 14 & 40 & 21 & 0 & đúng giờ\\
\hline
19 & 23 & 48 & 13 & 0 & đúng giờ\\
\hline
14 & 27 & 43 & 12 & 0 & đúng giờ\\
\hline
13 & 33 & 44 & 9 & 0 & đúng giờ \\
\hline
45 & 26 & 25 & 25 & 0 & trễ giờ\\
\hline
56 & 26 & 27 & 16 & 0 & trễ giờ\\
\hline
29 & 22 & 41 & 22 & 1 & trễ giờ\\
\hline
83 & 11 & 15 & 12 & 5 & trễ giờ\\
\hline
28 & 32 & 36 & 21 & 0 & trễ giờ\\
\hline
\end{tabular}
\end{flushleft}
Như đã nói phần trên, ta có $\mathrm{P}$(về đích đúng giờ) = 0.84 và $\mathrm{P}$(về đích trễ giờ) = 0.16
\section{Vẽ hàm mật độ xác suất của các biến}
Kham khảo \ref{density}, \ref{densityOverlap} tại mục Phụ lục, dùng giải thuật Kernel Density Estimation để vẽ hàm mật độ xác suất của các biến dưới đây:
\FloatBarrier
\begin{figure}[!htb]
\minipage{0.5\textwidth}
  \includegraphics[width=\linewidth]{DensityVerySmallStep_OnTime}
\endminipage
\minipage{0.5\textwidth}
  \includegraphics[width=\linewidth]{DensityVerySmallStep_LateTime}
\endminipage
\caption*{Hàm mật độ xác suất của biến $X_1$ bước đi 0-15 km/h}
\end{figure}
\FloatBarrier
Nếu gộp lại \\
\FloatBarrier
\begin{figure}[h]
    \begin{subfigure}[b]{0.7\textwidth}
        \includegraphics[width=\textwidth]{DensityVerySmallStep.png}
        \caption*{Hàm mật độ xác suất của biến $X_1$ bước đi 0-15 km/h}
    \end{subfigure}
\end{figure}
\FloatBarrier
\FloatBarrier
\begin{figure}[!htb]
\minipage{0.5\textwidth}
  \includegraphics[width=\linewidth]{DensitySmallStep_OnTime}
\endminipage
\minipage{0.5\textwidth}
  \includegraphics[width=\linewidth]{DensitySmallStep_LateTime}
\endminipage
\caption*{Hàm mật độ xác suất của biến $X_2$ bước đi 15-30 km/h}
\end{figure}
\FloatBarrier
Nếu gộp lại \\
\FloatBarrier
\begin{figure}[h]
    \begin{subfigure}[b]{0.7\textwidth}
        \includegraphics[width=\textwidth]{DensitySmallStep.png}
    \end{subfigure}
    \caption*{Hàm mật độ xác suất của biến $X_2$ bước đi 15-30 km/h}
\end{figure}
\FloatBarrier

\FloatBarrier
\begin{figure}[!htb]
\minipage{0.5\textwidth}
  \includegraphics[width=\linewidth]{DensityMediumStep_OnTime}
\endminipage
\minipage{0.5\textwidth}
  \includegraphics[width=\linewidth]{DensityMediumStep_LateTime}
\endminipage
\caption*{Hàm mật độ xác suất của biến $X_3$ bước đi 30-45 km/h}
\end{figure}
\FloatBarrier
Nếu gộp lại \\
\FloatBarrier
\begin{figure}[h]
    \begin{subfigure}[b]{0.7\textwidth}
        \includegraphics[width=\textwidth]{DensityMediumStep.png}
    \end{subfigure}
\caption*{Hàm mật độ xác suất của biến $X_3$ bước đi 30-45 km/h}
\end{figure}
\FloatBarrier

\FloatBarrier
\begin{figure}[!htb]
\minipage{0.5\textwidth}
  \includegraphics[width=\linewidth]{DensityHighStep_OnTime}
\endminipage
\minipage{0.5\textwidth}
  \includegraphics[width=\linewidth]{DensityHighStep_LateTime}
\endminipage
\caption*{Hàm mật độ xác suất của biến $X_4$ bước đi 45-60 km/h}
\end{figure}
\FloatBarrier
Nếu gộp lại \\
\FloatBarrier
\begin{figure}[h]
    \begin{subfigure}[b]{0.7\textwidth}
        \includegraphics[width=\textwidth]{DensityHighStep.png}
    \end{subfigure}
\caption*{Hàm mật độ xác suất của biến $X_4$ bước đi 45-60 km/h}
\end{figure}
\FloatBarrier
\FloatBarrier
\begin{figure}[!htb]
\minipage{0.5\textwidth}
  \includegraphics[width=\linewidth]{DensityVeryHighStep_OnTime}
\endminipage
\minipage{0.5\textwidth}
  \includegraphics[width=\linewidth]{DensityVeryHighStep_LateTime}
\endminipage
\caption*{Hàm mật độ xác suất của biến $X_5$ trên 60 km/h}
\end{figure}
\FloatBarrier
Nếu gộp lại \\
\FloatBarrier
\begin{figure}[h]
    \begin{subfigure}[b]{0.7\textwidth}
        \includegraphics[width=\textwidth]{DensityVeryHighStep.png}
    \end{subfigure}
\caption*{Hàm mật độ xác suất của biến $X_5$ trên 60 km/h}
\end{figure}
\FloatBarrier
\section{Tìm xác suất của điểm mới}
Nếu sử dụng nhân Gaussian trong thuật toán Kernel Density Estimation, ta có công thức tính xác suất của điểm mới 
\[
\hat{f}(x_i) = \hat{p}_\textit{\tiny\tiny KDE}(x_i) = \frac{1}{n}\sum_{i=1}^{n} K_h = \frac{1}{n}\frac{1}{h}\sum_{i=1}^{n} K(\frac{x-x_i}{h})
\]
Nếu muốn hiện thực bằng dòng lệnh R, kham khảo \ref{calculateNewPoint}
\section{Kết luận}
\underline{Bài toán}: Dự đoán xác suất xe buýt tuyến 72 lộ trình xuất phát từ BX Củ Chi về trạm đích BX An Sương đúng giờ (hạn mức 45 phút)\\
\underline{Cách giải}:
\begin{itemize}
\item Chỉ giữ lại các bước di chuyển trong 80\% thời gian đầu
\item Dữ liệu làm việc là dữ liệu sau khi được đồng bộ hóa khoảng cách thời gian hồi đáp (20 giây)
\item Bài toán này sẽ có 5 biến: 
\begin{itemize}
\item $X_1$: bước di chuyển 0-15 km/h
\item $X_2$: bước di chuyển 15-30 km/h
\item $X_3$: bước di chuyển 30-45 km/h
\item $X_4$: bước di chuyển 45-60 km/h
\item $X_5$: bước di chuyển 60 km/h
\end{itemize}
\item Ta chọn thời gian hoàn thành trước đúng 45 phút là về đích đúng giờ\\
Như vậy ta có $\mathrm{P}$(về đích đúng giờ) = 0.84 và $\mathrm{P}$(về đích trễ giờ) = 0.16
\item Dùng giải thuật Kernel Density Estimation để vẽ hàm mật độ xác suất của các biến
\item Sử dụng công thức Bayes tổng quát 
\[ %\mathlarger{
  \mathbb{P}(B_{i}|A)= \dfrac{\mathbb{P}(B_{i})\cdot \mathbb{P}(A|B_{i})}{\mathbb{P}(A)}
            =\dfrac{\mathbb{P}(B_{i})\cdot \mathbb{P}(A|B_{i})}{\sum_{j=1}^{m} \mathbb{P}(B_j) \cdot \mathbb{P}(A|B_j)}
%}
\]
các biến cố $A_{1}, A_{2},..., A_{n}$ là n biến cố \textbf{độc lập lẫn nhau} thì
\[
\mathbb{P}\left[\bigcap_{i=1}^n A_j \right] = \prod_{i=1}^n \mathbb{P}[A_i]
\]
\end{itemize}

\chapter{KẾT LUẬN}
\section{Tổng kết}
Trình bày của chương mở đầu đã nói hết cho phần tổng kết. Tác giả nhắc lại đề tài Luận Văn: Dùng Thống Kê định lượng kinh nghiệm di chuyển trên một lộ trình quen thuộc, dự đoán xác suất xe buýt về trạm đích đúng giờ. Chương mở đầu đã trình bày cách nghiên cứu bài toán và nêu sơ lược các bước trong phương pháp giải. Các chương tiếp theo để khẳng định sử dụng xác suất Bayes là hợp lý để giải bài toán tìm xác suất xe buýt về trạm đúng giờ. 
\section{Đóng góp của đề tài}
Tác giả không đặt mục tiêu tham vọng giải bài toán này có thể đem ra ứng dụng thực tế được vì bản thân tác giả không phải là nhà thống kê, không có nhiều kinh nghiệm làm việc với kích thước mẫu rất nhỏ có thể đưa ra dự đoán đúng trên kích thước quần thể. Nhưng thông qua Luận Văn, tác giả chứng minh được sử dụng kiến thức Thống Kê cơ bản (không chứng minh công thức Toán học vì tác giả không phải là nhà Toán học), giải được bài toán của Luận Văn.
\section{Hướng phát triển}
Tác giả không muốn đề cập cách giải bài toán của Luận Văn với dữ liệu cực lớn mà muốn đề cập đến sử dụng kiến thức Thống Kê giải với dữ liệu mẫu ngẫu nhiên, kích thước rất nhỏ. Với phương châm như vậy, tác giả chia sẽ một số phát hiện để nếu có ai hứng thú muốn phát triển đề tài của Luận Văn này. Tác giả đã phát hiện ngoài sử dụng phương pháp xác suất Bayes để phân loại, có rất nhiều thuật toán khác trong Thống Kê để phân loại. Ví dụ như thuật toán FRBCS.W giúp phân loại dữ liệu của tác giả đúng đến 75\% (nhưng tác giả không trình bày bằng chứng ở đây vì nó nằm ngoài mục tiêu trình bày sử dụng phương pháp Bayes của Luận Văn). Còn muốn tìm thời gian xe buýt về trạm đích khi học các dữ liệu mẫu giới hạn trong quá khứ thì tác giả tìm được các thuật toán ước lượng hồi quy trong Thống Kê như thuật toán Wang and Mendel, giúp ta làm điều này, nhưng tác giả cũng không trình bày ở đây.\\
Nếu có tham vọng muốn bài toán của Luận Văn này được triển khai trên thực tế, thì người đọc cần bổ sung thêm kiến thức Thống Kê như làm thế nào tạo ra thiết kế thực nghiệm, tiêu chuẩn lựa chọn mô hình để rút ra kết luận có tính thuyết phục hơn.
\pagebreak
\chapter{PHỤ LỤC}
\begin{enumerate}[label=\textbf{PL\arabic*}]
\item \label{histogramWithNormalCurve} Mã nguồn R vẽ histogram dữ liệu
\begin{flushleft}
\begin{tabular}{  |l| }
\hline 
\textit{\#set your working directory where your file data locates}\\
setwd("/home/thuy1/git/predictUsingProbability/Preprocess")\\
data=read.table(file="CC\_AS\_FinishTime.csv")[,1]\\
hist(data,breaks=12, prob=TRUE,\\
\hspace{1cm} xlab="Thời gian hoàn thành BX Củ Chi đến BX An Sương",\\
\hspace{1cm} main="Biểu đồ phân phối thời gian hoàn thành $\textbackslash$n BX Củ Chi đến BX An Sương")\\
curve(dnorm(x, mean=mean(data), sd=sd(data)), add=TRUE)\\
\hline
\end{tabular}
\end{flushleft}
\item  \label{trucquandichuyen} Mã nguồn R trực quan hóa giá trị các bước di chuyển của 24 chuyến xe trên\\
\begin{flushleft}
\begin{tabular}{ |l| }
\hline 
\textit{\#set your working directory where your file data locates}\\
setwd("/home/thuy/workspace/Preprocess")\\
\textit{\#count maximum column length}\\
count.fields("data.txt", sep = ",")\\
maxCol <- max(count.fields("data.txt", sep = ","))\\
\textit{\#load data with different column length}\\
dat=read.table("data.txt", header = FALSE, \\
\hspace{2cm} col.names = 1:maxCol, \textit{\#maxCol is maximum column length in your data row}\\ 				
\hspace{2cm} sep = ",",\\ 
\hspace{2cm} fill = TRUE) \textit{\#set value NA for empty column}\\
i=1\\
for (i in 1:nrow(dat)) \{ \\
  \hspace{1cm} d=dat[i,] \textit{\# get row data}\\
  \hspace{1cm} d=d[!is.na(d)] \textit{\#remove column NA}\\
  \hspace{1cm} x=length(d)\\
  \hspace{1cm} \textit{\#capture image}\\
  \hspace{1cm} png(filename=paste("capture", i, ".png", sep = ""))\\
  \hspace{1cm} \textit{\#remove axes}\\
  \hspace{1cm} temp <- plot(1:x, d, type='b', axes=FALSE, xlab = "step", ylab = "length (meters)")\\
  \hspace{1cm} \textit{\#adjust axes length}\\  
  \hspace{1cm} temp <- axis(side=1, at=c(1:x))\\
  \hspace{1cm} temp <- axis(side=2, at=seq(min(d), max(d), by=100))\\
  \hspace{1cm} temp <- box()\\
  \hspace{1cm} print(temp)\\
  \hspace{1cm} dev.off()\\
  \}
\\
\hline
\end{tabular}
\end{flushleft}
\item \label{xacsuatTLTGHT} Mã nguồn R xuất hình thể hiện xác suất tích lũy thời gian hoàn thành BX Củ Chi - BX An Sương\\
\begin{flushleft}
\begin{tabular}{ |l| }
\hline
\textit{\#set your working directory where your file data locates}\\
setwd("/home/thuy/workspace/Preprocess")\\
y = read.csv("CC\_AS\_Rep.csv",header=FALSE)\$V1\\
p = ecdf(y)\\
plot(p,\\
\hspace{1cm} xlab = 'Thời gian hoàn thành', \\
\hspace{1cm} ylab = 'Xác suất tích lũy', \\
\hspace{1cm} main = 'Xác suất tích lũy thời gian hoàn thành từ BX Củ Chi đến BX An Sương'
)\\
abline(v = 45, h = 0.83773583,col="red",lwd=2, lty=2)\\
legend(45, 0.83773583, '84\% tại điểm 45', box.lwd = 0)\\
abline(v = 48, h = 0.9554717,col="blue",lwd=2, lty=2)\\
legend(48, 0.9554717, '96\% tại điểm 48', box.lwd = 0)\\ 
\hline
\end{tabular}
\end{flushleft}
\item \label{sapxepTGHT} Mã nguồn R sắp xếp tăng dần xác suất thời gian hoàn thành BX Củ Chi - BX An Sương\\
\begin{flushleft}
\begin{tabular}{ |l| }
\hline
\textit{\#set your working directory where your file data locates}\\
setwd("/home/thuy1/git/predictUsingProbability/Preprocess/")\\
mydata = read.csv("CC\_AS\_Freq.csv",sep = "|",header=FALSE)[ ,1:2]\\
\textit{\#set column name for your data}\\
colnames(mydata) <- c("X1","X2")\\
X1=mydata\$X1\\
X2=mydata\$X2\\
mydata\$X1 <- factor(mydata\$X1, levels = mydata\$X1[order(mydata\$X2)])\\
library(ggplot2)\\
ggplot(mydata, aes(x = mydata\$X1, y = mydata\$X2)) +\\
\hspace{1cm} theme\_bw() + geom\_bar(stat = "identity") + \\
\hspace{1cm} xlab("Thời gian hoàn thành ") +\\
\hspace{1cm} ylab("Tần số xuất hiện") \\
\hline
\end{tabular}
\end{flushleft}

\item \label{density} Mã nguồn R vẽ đường density chồng lên histogram
\begin{flushleft}
\begin{tabular}{  |l| }
\hline 
\textit{\#set your working directory where your file data locates}\\
setwd('/home/thuy1/git/predictUsingProbability/Preprocess')\\
lateTime=read.table(file="freqVerySmall\_LateTime.csv")[,1]\\
hist(lateTime, probability = TRUE,\\
     main="the Gaussian smoothing kernel of VerySmall Step $\textbackslash$n belonged to class LateTime",\\
     ylab="density", xlab="the occurrences of VerySmall Step"\\
     )\\
lines(density(lateTime, kernel=c("gaussian")), col="blue")\\
legend("topright", inset=.05, c("lateTime"), fill=c("blue"), horiz=TRUE)\\
onTime=read.table(file="freqVerySmall\_OnTime.csv")[,1]\\
hist(onTime, probability = TRUE,\\
     main="the Gaussian smoothing kernel of VerySmall Step $\textbackslash$n belonged to class OnTime",\\
     ylab="density", xlab="the occurrences of VerySmall Step"\\
     )
lines(density(onTime, kernel=c("gaussian")), col="green")\\
legend("topright", inset=.05, \\
       c("onTime"), fill=c("green"), horiz=TRUE)\\
\hline
\end{tabular}
\end{flushleft}

\item \label{densityOverlap} Mã nguồn R vẽ kết quả đồ họa hai hàm density chồng lên nhau
\begin{flushleft}
\begin{tabular}{  |l| }
\hline 
\textit{\#set your working directory where your file data locates}\\
setwd('/home/thuy1/git/predictUsingProbability/Preprocess')\\
\textit{\#load the occurrences of very high step length belonged to class onTime}\\
onTime=read.table(file="freqVeryhigh\_OnTime.csv")[,1]\\
density(onTime, kernel=c("gaussian"))\\
plot(density(onTime, kernel=c("gaussian")),ylim=c(0.0, 1),\\
\hspace{.5cm} main="Density of the occurrences of very high step\\
\hspace{1.7cm} length $\textbackslash$n using the Gaussian smoothing kernel",\\
\hspace{.5cm}     ylab="density", xlab="the occurrences of very high step length",\\
\hspace{.5cm}     col="green")\\
\textit{\#load the occurrences of very high step length belonged to class lateTime}\\
lateTime=read.table(file="freqVeryhigh\_LateTime.csv")[,1]\\
density(lateTime, kernel=c("gaussian"))\\
lines(density(lateTime, kernel=c("gaussian")),col="yellow")\\
\textit{\#load add comment into the picture}\\
legend("topright", inset=.05, title="density",\\
\hspace{.7cm} c("onTime","lateTime"), fill=c("green","yellow"), horiz=TRUE)\\
\hline
\end{tabular}
\end{flushleft}

\item \label{calculateNewPoint} Với xs là các điểm lấy mẫu, h là bandwidth, viết hàm tính myKDE cho điểm mới t\\
\begin{tabular}{ |l| } 
 \hline
d <- density(xs)\\
h = d\$bw\\
myKDE <- function(t)\{\\
\hspace{1cm}kernelValues <- rep(0,length(xs))\\
\hspace{1cm}for(i in 1:length(xs))\{\\
\hspace{2cm}transformed = (t - xs[i]) / h\\
\hspace{2cm}kernelValues[i] <- dnorm(transformed, mean = 0, sd = 1) / h\\
\hspace{1cm}\}\\
\hspace{1cm}return(sum(kernelValues) / length(xs))\\
\}\\
\hline
\end{tabular}
\item Mã nguồn R chạy giải thuật FRBCS.W
\begin{flushleft}
\begin{tabular}{  |l| }
\hline 
library(frbs)\\
\textit{\#set your working directory where your file data locates}\\
setwd('/home/thuy1/git/predictUsingProbability/Preprocess')\\
data=read.table(file="classifyRoute.csv", header=FALSE, sep=",",\\
\hspace{1cm} col.names = c("X1","X2","X3","X4","X5","Clazz"));\\
\textit{\#The dataset is shuffled}\\
dataShuffled <- data[sample(nrow(data)),]\\
\textit{\#the last column is the output variable/attribute and it must be expressed in numbers (numerical data)}\\
dataShuffled[,6] <- unclass(dataShuffled[,6])\\
\textit{\#dataset divided into training and testing data}\\
\textit{\#100 first records is training data}\\
tra.data <- dataShuffled[1:100,]\\
\textit{\#90 last records is testing data and their output variable/attribute is removed}\\
tst.data <- dataShuffled[100:nrow(dataShuffled),1:5]\\
\textit{\#make the output variable/attribute of testing data an 1-column matrix}\\
real.data <- matrix(dataShuffled[100:nrow(dataShuffled),6], ncol = 1)\\
\textit{\#return the range [min, max] of every input variable/attribute}\\
range.data.input <- matrix(apply(data[, -ncol(data)], 2, range), nrow = 2)\\
range.data.input\\
\textit{\#\hspace{1cm}[,1]\hspace{0.5cm}[,2]\hspace{0.5cm}[,3]\hspace{0.5cm}[,4]\hspace{0.5cm}[,5]}\\
\textit{\#[1,]\hspace{0.5cm}4\hspace{0.8cm}8\hspace{0.8cm}15\hspace{0.8cm}3\hspace{0.8cm}0}\\
\textit{\#[2,]\hspace{0.5cm}83\hspace{0.6cm}74\hspace{0.7cm}68\hspace{0.7cm}35\hspace{0.7cm}11}\\
\textit{\# Set the method and its parameters. In this case we use FRBCS.W algorithm}\\
method.type <- "FRBCS.W"\\
control <- list(num.labels = 7, type.mf = "GAUSSIAN", type.tnorm = "MIN",\\
\hspace{2.3cm}  type.snorm = "MAX", type.implication.func = "ZADEH")\\
\textit{\# Learning step: Generate fuzzy model}\\
object.cls <- frbs.learn(tra.data, range.data.input, method.type, control)\\
\textit{\# Predicting step: Predict newdata}\\
res.test <- predict(object.cls, tst.data)\\
\textit{\# Display the FRBS model}\\
summary(object.cls)\\
\textit{\# Plot the membership functions}\\
plotMF(object.cls)\\
\hline
\end{tabular}
\end{flushleft}
Hình vẽ thể hiện giải thuật FRBCS.W
\FloatBarrier
\begin{figure}[h!]
        \begin{subfigure}[b]{0.7\textwidth}
        		\label{tab:example2}
                \includegraphics[width=\linewidth]{FRBCSW}
        \end{subfigure}%
\end{figure}
\FloatBarrier
\item \label{DragAndDrop} Ứng dụng bản demo "Drag And Drop data layer GeoJSON" \footnote{https://developers.google.com/maps/documentation/javascript/examples/layer-data-dragndrop} của Google Map API để kiểm tra lộ trình\\
\FloatBarrier
\begin{figure}[h]
\caption{Ứng dụng bản demo "Drag And Drop data layer GeoJSON"}
    \begin{subfigure}[b]{0.4\textwidth}
        \includegraphics[width=\textwidth]{dragAndDrop1.png}
        \caption{Thả file có định dạng lớp dữ liệu GeoJSon vào bản đồ}
    \end{subfigure}
    \begin{subfigure}[b]{0.4\textwidth}
        \includegraphics[width=\textwidth]{dragAndDrop2.png}
        \caption{Lớp dữ liệu GeoJson vừa thả xuất hiện trên bản đồ}
    \end{subfigure}
\end{figure}
\FloatBarrier

\item Cách thức kết nối R với Java\\
Thiết lập trên hệ điều hành Ubuntu:
\begin{itemize}
\item Cài Oracle JDK 8:\\
\begin{flushleft}
\begin{tabular}{  |l| }
\hline 
\hspace{.1cm} sudo add-apt-repository ppa:webupd8team/java\\
\hspace{.1cm} sudo apt-get install oracle-java8-installer\\
\hspace{.1cm} sudo update-alternatives --config java\\
\textit{\#Chọn đường dẫn Java mặc định /usr/lib/jvm/java-8-oracle/jre/bin/java}\\
\hline
\end{tabular}
\end{flushleft}
\item Cài môi trường R\\
\begin{flushleft}
\begin{tabular}{  |l| }
\hline 
\hspace{.1cm} sudo apt-get install r-base\\
\textit{\#cài đặt công cụ lập trình R, ví dụ RStudio, sau đó chạy các câu lệnh sau}\\
\hspace{.1cm} install.packages("rJava")\\
\textit{\#Nếu bước cài đặt môi trường JDK không đúng, sẽ báo lỗi}\\
\textit{\#"configure: error: Cannot compile a simple JNI program.}\\
\textit{\#Make sure you have Java Development Kit installed and correctly registered in R."}\\
\hspace{.1cm} R.home()\\ 
\textit{\#Lấy đường dẫn của R\_HOME}\\
\hline
\end{tabular}
\end{flushleft}
\item Thiết lập kết nối R sử dụng công cụ lập trình Eclipse
\begin{flushleft}
\begin{tabular}{  |l| }
\hline 
\textit{\#Thêm thư viện R, sử dụng "Configure Build Path"}\\
\hspace{.1cm} Thêm /home/thuy1/R/i686-pc-linux-gnu-library/3.3/rJava/jri/JRI.jar \\
\hspace{.1cm} vào tab Libraries\\
\textit{\#Trước khi chạy cần phải thiết lập biến môi trường kết nối đến R,}\\
\textit{\#sử dụng Run Configuration}\\
\hspace{.1cm} Thêm -Djava.library.path=/home/thuy1/R/i686-pc-linux-gnu-library/3.3/rJava/jri/ \\
\hspace{.1cm} vào tab VM Arguments\\
\hspace{.1cm} Thêm R\_HOME vào tab Environment\\
\hline
\end{tabular}
\end{flushleft}


  
\end{itemize}
 
\item \label{tinhkhoangcach} Mã nguồn Java tính khoảng cách giữa 2 tọa độ
\begin{flushleft}
\begin{tabular}{ |l| }
\hline
public static float distFrom(double lat1, double lng1, double lat2, double lng2) \{\\
\hspace{0.5cm} double earthRadius = 6371000; //meters\\
\hspace{0.5cm} double dLat = Math.toRadians(lat2-lat1);\\
\hspace{0.5cm} double dLng = Math.toRadians(lng2-lng1);\\
\hspace{0.5cm} double a = Math.sin(dLat/2) * Math.sin(dLat/2) +\\
\hspace{2.3cm} Math.cos(Math.toRadians(lat1)) * Math.cos(Math.toRadians(lat2)) *\\
\hspace{2.3cm} Math.sin(dLng/2) * Math.sin(dLng/2);\\
\hspace{0.5cm} double c = 2 * Math.atan2(Math.sqrt(a), Math.sqrt(1-a));\\
\hspace{0.5cm} float dist = (float) (earthRadius * c);\\
\hspace{0.5cm} return dist;\\
\}\\
\hline
\end{tabular}
\end{flushleft}

\item \label{dongbohoa20} Mã nguồn Java tính gom nhóm theo chuẩn 20 giây 
\begin{flushleft}
\begin{tabular}{ |l| }
\hline
function filterStandardResponse(List<Long> responseDurationList) \{\\
\hspace{0.5cm} List<Long> nonStandardList = new ArrayList<Long>();\\
\hspace{0.5cm} for (Long duration: responseDurationList) \{\\
\hspace{1cm} if (duration \% 20 == 0) \{\\
\hspace{1.5cm} break2StandardResponse(duration);\\
\hspace{1cm} \} else \{\\
\hspace{1.5cm} nonStandardList.add(duration); \\
\hspace{1cm} \}\\
\hspace{0.5cm} \}\\
\hspace{0.5cm} sort(nonStandardList);\\
\hspace{0.5cm} accumulate2ElmntToStandardResponse();\\
\}\\
\hline
\end{tabular}
\end{flushleft}
\begin{flushleft}
\begin{tabular}{ |l| }
\hline
function accumulate2ElmntToStandardResponse(List<Long> sortedNonStandardDurationList) \{\\
\hspace{0.5cm} List<Long> blockIndexList = new ArrayList<Long>();\\
\hspace{0.5cm} int maxIndex = sortedNonStandardDurationList.size();\\
\hspace{0.5cm} for (int i = 0; i <= maxIndex - 2; i++) \{\\
\hspace{1cm}   for (int j = 0; j <= maxIndex - 1; j++) \{\\
\hspace{1.5cm} long duration1 = sortedDuration.get(i); \\
\hspace{1.5cm} long duration2 = sortedDuration.get(j);\\
\hspace{1.5cm} long sumDuration = duration1+duration2;\\
\hspace{1.5cm} if (sumDuration \% 20 == 0 \&\& !blockList.contains(i) \&\& !blockList.contains(j)) \{\\
\hspace{2cm} blockList.add(i);\\
\hspace{2cm} blockList.add(j);\\
\hspace{2cm} break2StandardResponse(duration);\\
\hspace{1.5cm} \}\\
\hspace{1cm} \}\\
\hspace{0.5cm} \}\\
\hspace{0.5cm} List<Long> newSortedNonStandardList = new ArrayList<Long>();\\
\hspace{0.5cm} for (int i = 0; i < maxIndex; i++) \{\\
\hspace{1.5cm}     if (!blockList.contains(i)) \{\\
\hspace{2cm}        newSortedNonStandardList.add(sortedNonStandardDurationList.get(i));\\
\hspace{1.5cm}     \}\\
\hspace{0.5cm} \}\\
\hspace{0.5cm} if (newSortedNonStandardList.size() >= 3) \{\\
\hspace{1.5cm}      accumulate3ElmntToStandardResponse(newSortedNonStandardList);\\
\hspace{0.5cm} \}\\
\hline
\end{tabular}
\end{flushleft}
Các hàm accumulate3ElmntToStandardResponse và accumulate4ElmntToStandardResponse tương tự như hàm accumulate2ElmntToStandardResponse để thực hiện gom nhóm 3 thành phần, 4 thành phần thành bội số 20, cuối cùng với những thành phần chưa được gom nhóm. Tiếp theo, thực hiện việc gom nhóm các thành phần liên tiếp thành bội số 20. Cuối cùng, thực hiện gom nhóm các thành phần liên tiếp sao cho chia cho 20 có số dư nhỏ nhất\\
\begin{flushleft}
\begin{tabular}{ |l| }
\hline
function accumulateSequenceElmtToStandardResponse(List<Long> sortedNonStandardDurationList) \{\\
\hspace{0.5cm} int i = 0;\\
\hspace{0.5cm} int j = 1;\\
\hspace{0.5cm} int maxIndex = sortedNonStandardDurationList.size();\\
\hspace{0.5cm} while (i<maxIndex) \{\\
\hspace{1cm}      long sumDuration = sortedDuration.get(i);\\
\hspace{1cm}      while (j < maxIndex) \{\\
\hspace{1.5cm}       sumDuration += sortedDuration.get(j);\textit{\#sum of continuous elements}\\
\hspace{1.5cm}       if (sumDuration\%20==0) \{\\
\hspace{2cm}           break2StandardResponse(sumDuration);\\
\hspace{2cm}           i=j+1;\textit{\#create new caculation with start index equal j+1}\\
\hspace{2cm}           j=i+1;\\
\hspace{2cm}           break;\\
\hspace{1.5cm}       \} else \{\\
\hspace{2cm}           j++;\textit{\#continue until sumDuration\%20 equal 0}\\
\hspace{1.5cm}        \}\\
\hspace{1cm}       \}\\
\hspace{1cm}     if (j == maxIndex) \{\\
\hspace{1.5cm}        break;\\
\hspace{1cm}     \}\\
\hspace{0.5cm} \}\\
\hspace{0.5cm} List<Long> notAccumulateElmtList = new ArrayList<Long>();\\
\hspace{0.5cm} while (i<maxIndex) \{\\
\hspace{1cm}      notAccumulateElmtList.add(sortedNonStandardDurationList.get(i));\\
\hspace{1cm}      i++;\\
\hspace{0.5cm} \}\\
\hspace{0.5cm} accumulateSequenceElmtWithMinRedundant(notAccumulateElmtList);\\
\hline
\end{tabular}
\end{flushleft}

\begin{flushleft}
\begin{tabular}{ |l| }
\hline
function accumulateSequenceElmtWithMinRedundant(List<Long> sortedNonStandardDurationList) \{\\
\hspace{0.5cm} int i = 0;\\
\hspace{0.5cm} int j = 1;\\
\hspace{0.5cm} int stopIndex = 0;\\
\hspace{0.5cm} int maxIndex = sortedNonStandardDurationList.size();\\
\hspace{0.5cm} while (i<maxIndex) \{\\
\hspace{1cm} long sumDuration = sortedNonStandardDurationList.get(i);\\
\hspace{1cm} stopIndex = i;\\
\hspace{1cm} long compareNumber = 20;\\
\hspace{1cm} if (sumDuration/20>0) \{\textit{\#if sumDuration > 20}\\
\hspace{1.5cm}   compareNumber = 20*(sumDuration/20);\\
\hspace{1cm} \}\\
\hspace{1cm} long min = Math.abs(sumDuration - compareNumber);\\
\hspace{1cm} while (j < maxIndex) \{\\
\hspace{1.5cm}   sumDuration += sortedNonStandardDurationList.get(j);\\
\hspace{1.5cm}   long currentCompareNumber = 20;\\
\hspace{1.5cm}   if (sumDuration/20>0) \{\textit{\#if sumDuration > 20}\\
\hspace{2cm}          currentCompareNumber = 20*(sumDuration/20);\\
\hspace{1.5cm}   \}\\
\hspace{1.5cm}   long currentMin = Math.abs(sumDuration - currentCompareNumber);\\
\hspace{1.5cm}   if (currentMin <= min) \{\\
\hspace{2cm}       stopIndex = j;\\
\hspace{2cm}       min = currentMin;\\
\hspace{1.5cm}    \}\\
\hspace{1.5cm}   j++;\\
\hspace{1cm} \}\\
\hspace{1cm}  if (stopIndex != i) \{\\
\hspace{1.5cm}     sumDuration = 0;\\
\hspace{1.5cm}     for (int k = i; k <= stopIndex; k++) \{\\
\hspace{2cm}         sumDuration += sortedNonStandardDurationList.get(k);\\
\hspace{1.5cm}     \}\\  
\hspace{1.5cm}     break2StandardResponse(sumDuration);  \\            
\hspace{1cm}   \}\\
\hspace{1cm}     i = stopIndex + 1;\\
\hspace{1cm}     j = i + 1;\\
\hspace{0.5cm} \}\\
\hline
\end{tabular}
\end{flushleft}


\end{enumerate}


\clearpage
\newpage
\begin{center}{\fontsize{16pt}{1}\selectfont \textbf{LÝ LỊCH TRÍCH NGANG}}\\\end{center}
\vspace*{0.1cm}Họ và tên: Lê Thị Minh Thùy \\
\vspace*{0.1cm}Ngày sinh: 22/01/1986 \\
\vspace*{0.1cm}Nơi sinh: Đồng Nai \\
\vspace*{0.1cm}Địa chỉ liên lạc: 741 Trương Công Định Phường 9, TP Vũng Tàu \\
\vspace*{0.1cm}Email: thuyltm2201@gmail.com \\
\vspace*{0.1cm}\textbf{QUÁ TRÌNH ĐÀO TẠO}\\
\vspace*{0.5cm}\begin{tabular}{ |c|c|c|c| } 
\hline
 Thời gian & Trường đào tạo & Chuyên ngành & Trình độ đào tạo\\
\hline
 2004 – 2009 & Trường Đại học Bách Khoa TP.HCM & Công nghệ thông tin & Cử nhân\\
\hline 
 2013 – 2017 & Trường Đại học Bách Khoa TP.HCM & Khoa học máy tính & Thạc sĩ\\
\hline
\end{tabular}\\
\vspace*{0.5cm}\textbf{QUÁ TRÌNH CÔNG TÁC}\\
\vspace*{0.5cm}\begin{tabular}{ |c|p{9.8cm}|c| } 
\hline
 Thời gian & Đơn vị công tác & Chuyên ngành\\
\hline
 2014 – 2016 & Công ty gia công phần mềm Tường Minh & Lập trình viên\\
\hline 
\end{tabular}

\cleardoublepage
\addcontentsline{toc}{chapter}{DANH MỤC KHAM KHẢO}
\begin{thebibliography}{9}
\section*{Tài liệu trong nước}
\bibitem{TKCNUDR}	
	Người dịch: Nguyễn Văn Minh Mẫn, 
	\emph{Thống kê Công nghiệp hiện đại với ứng dụng viết trên R, MINITAB và JMP},
	Nhà xuất bản Bách Khoa Hà Nội,2016, pp.19-131
\section*{Tài liệu nước ngoài}
\bibitem{Brockwell_intro}
	Jiawei Han, Micheline Kamber, Jian Pei (2012),
	\emph{Data Mining: Concepts and Techniques (3rd ed.)},
	Morgan Kaufmann Publishers, USA.
\section*{Website}
\bibitem{2WAYSOFNB}
	\emph{2 ways of using Naive Bayes classification for numeric attributes}, truy cập ngày 1 tháng 3 năm 2017,
	địa chỉ \emph{http://www.simafore.com/blog/bid/107702/2-ways-of-using-Naive-Bayes-classification-for-numeric-attributes}..
\bibitem{Gaussian classifier}
	\emph{The Gaussian classifier}, truy cập ngày 2 tháng 3 năm 2017,
	địa chỉ \emph{http://www.svcl.ucsd.edu/courses/ece271A/handouts/GC.pdf}.
\bibitem{Gaussian classifier}
	\emph{Naive Bayes 3: Gaussian example}, truy cập ngày 2 tháng 3 năm 2017,
	địa chỉ \emph{https://www.youtube.com/watch?v=r1in0YNetG8}.	
\bibitem{DKE1}
	\emph{L7: Kernel density estimation}, truy cập ngày 26 tháng 3 năm 2017,
	địa chỉ \emph{http://www.stat.washington.edu/courses/stat539/spring13/Handouts/tamu-csce-666-pr\_l7-annotated.pdf}.	
\bibitem{DKE2}
	\emph{Kernel density estimation Wikipedia}, truy cập ngày 26 tháng 3 năm 2017,
	địa chỉ \emph{https://en.wikipedia.org/wiki/Kernel\_density\_estimation}.
\bibitem{DKE3}
	\emph{Exploratory Data Analysis: Kernel Density Estimation in R on Ozone Pollution Data in New York and Ozonopolis}, truy cập ngày 26 tháng 3 năm 2017,
	địa chỉ \emph{https://www.r-bloggers.com/exploratory-data-analysis-kernel-density-estimation-in-r-on-ozone-pollution-data-in-new-york-and-ozonopolis/}.
\bibitem{FPDNPU}
	\emph{Find-the-probability-density-of-a-new-data-point-using-density-function-in-r}, truy cập ngày 4 tháng 6 năm 2017,
	địa chỉ \emph{https://stackoverflow.com/questions/28077500/find-the-probability-density-of-a-new-data-point-using-density-function-in-r}
\section*{Developer's Website}
\bibitem{GeoJSON}
	\emph{Google Maps 3 API - Data Layer: GeoJSON}, truy cập ngày 6 tháng 11 năm 2016,
	địa chỉ \emph{https://developers.google.com/maps/documentation/javascript/examples/layer-data-style}.	
\bibitem{DADG}
	\emph{Google Maps 3 API - Click on feature (from geojson)}, truy cập ngày 6 tháng 11 năm 2016,
	địa chỉ \emph{http://stackoverflow.com/questions/29309856/google-maps-3-api-click-on-feature-from-geojson-and-check-if-it-contains-loc}.
\bibitem{DADG}
	\emph{Google Maps 3 API - Data Layer: Drag and Drop GeoJSON}, truy cập ngày 6 tháng 11 năm 2016,
	địa chỉ \emph{https://developers.google.com/maps/documentation/javascript/examples/layer-data-dragndrop}.		
\bibitem{WID}
	\emph{Google Maps 3 API - Waypoints in directions}, truy cập ngày 6 tháng 11 năm 2016,
	địa chỉ \emph{https://developers.google.com/maps/documentation/javascript/examples/directions-waypoints}.	
\bibitem{DM}
	\emph{Google Maps 3 API - Distance Matrix}, truy cập ngày 6 tháng 11 năm 2016,
	địa chỉ \emph{https://developers.google.com/maps/documentation/javascript/examples/distance-matrix}.	
\end{thebibliography}
\pagebreak
\end{document}

